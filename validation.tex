%!TEX root = reference.tex
\chapter{Validation Rules}
\label{validation}
\index{validation!rules}

In addition to macro replacement, it is also possible to specify one or more \emph{validation} rules to apply. Validation rules are applied by the compiler during parsing and are used to determine if the program is `well-formed' or not.


\section{Validation Rule}
\label{ValidationRule}
A validation rule allows the extension builder to define legal instances of elements of an extension. A validation rule defines the valid forms of expressions, statements or other program elements and also defines expectations for contained components.

A validation rule that expresses the constraints for elements in a \q{let} \q{in} form might be:
\begin{alltt}
# let ?Body use in ?Exp :: expression
  :- Body::statement :& Exp::expression;
\end{alltt}
This states that a \q{let}\ldots\q{in} form is a valid \q{expression}, provided that the \q{Body} is a valid instance of \q{statements} and \q{Exp} is a valid expression.

If a particular extension does not have preconditions, then the right hand side of the validation rule can be omitted. For example, 
\begin{alltt}
#natural h :: time;
\end{alltt}
states that a natural integer literal (a non-zero integer literal) followed by the \q{h} operator is a valid instance of a \q{time} expression.

The general form of a validation rule is shown in Figure~\vref{validationRuleFig}.
\begin{figure}[htbp]
\begin{eqnarray*}
\emph{MetaRule}&\arrowplus&\emph{ValidationRule}\\
\emph{ValidationRule}&\arrow&\q{\#} \emph{MetaPattern} \q{::} \emph{Category} \q{:-} \emph{Validation}\\
\emph{Validation}&\arrow&\emph{MetaExp} :: \emph{Category}\\
&\choice&\emph{ValidationConjunction}\\
&\choice&\emph{ValidationDisjunction}\\
&\choice&\emph{ValidationNegation}\\
&\choice&\emph{ValidationConditional}\\
&\choice&\emph{ValidationTupleIteration}\\
&\choice&\emph{ValidationBraceIteration}\\
&\choice&\emph{ValidationMessage}\\
&\choice&\emph{SubValidation}\\
\end{eqnarray*}
\caption{Validation Rules}
\label{validationRulesFig}
\end{figure}
where \q{\emph{MetaPattern}} is a validation pattern, \emph{Category} is a category.

\begin{aside}
If the pattern of a validation rule `fires' then \emph{no other} validation rule will be considered. That means that the condition part of the rule must account for all valid variations on the matched term.
\end{aside}

The validation category is denoted by an identifier -- any (non operator) identifier may be used. However, certain identifiers denote \emph{standard categories} -- categories that are defined by the language. See Section~\vref{standardCategories} for a listing of these.

In particular, the top-level category of a complete program package is always \q{Package}.

\subsection{Validation Patterns}

\subsubsection{The \q{identifier} pattern}
The pattern \q{identifier} matches an abstract syntax term if it is an identifier -- any identifier.

\subsubsection{The variable pattern}
A pattern of the form 
\begin{alltt}
?V
\end{alltt}
matches any term, and binds the rule variable \q{V} to it. The variable may be referenced in the condition part of the validation rule -- with or without the \q{?} prefix.

A pattern of the form
\begin{alltt}
\emph{Ptn}?\emph{V}
\end{alltt}
matches the term if the \q{\emph{Ptn}} matches the term, and it binds the rule variable \q{\emph{V}} to the matched term.

\subsubsection{The \q{identifier} pattern}
The \q{identifier} pattern matches any identifier.
\begin{aside}
The \q{identifier} pattern \emph{does not} match against standard keywords. To match against any identifier, keyword or not, use the \q{symbol} pattern.
\end{aside}

\subsubsection{The \q{symbol} pattern}
The \q{symbol} patterns matches any identifier -- including operators and keywords.

\subsubsection{The \q{string} pattern}
The \q{string} pattern matches a string literal.

\subsubsection{The \q{integer} pattern}
The \q{integer} pattern matches a positive literal integer. Note that this \emph{does not} match a negative integer literal.

\subsubsection{The \q{float} pattern}
The \q{float} pattern matches a positive literal floating point number.

\subsubsection{Literal terms}
Any term that is not one of the standard matching patterns is interpreted as a literal value -- unless it has argument terms that are interpreted.

For example, the pattern
\begin{alltt}
select ?X from ?P in ?L
\end{alltt}
matches any term that is formed by combining a \q{select} operator, a \q{from} operator, and an \q{in} operator in the data.
\begin{aside}
The identifiers \q{from}, and \q{in} are standard \emph{operators} -- see Section~\vref{standardOperators}. In order to permit this form we also need to declare \q{select} as an operator:
\begin{alltt}
#prefix((select),1150);
\end{alltt}
The above pattern is then equivalent to:
\begin{alltt}
select(from(?X,in(?P,?L)))
\end{alltt}
\end{aside}


\subsubsection{Special Validation Patterns}
The validation pattern:
\begin{alltt}
?F@?Arg
\end{alltt}
matches against any applicative term, and 'binds' the variable \q{?F} to the function operator in the applicative expression and \q{?Arg} to the argument(s) of the expression.

\subsection{Validation Conditions}
The body of a validation rule denotes a combination of \emph{validation conditions} -- conditions that the matched segment of the program must satisfy.

\subsubsection{Category Condition}
A condition of the form:
\begin{alltt}
\emph{Exp} :: \emph{Category}
\end{alltt}
means that the term identified by \q{\emph{Exp}} should satisfy the \q{\emph{Category}}. \q{\emph{Category}} is named with an identifier, and \q{\emph{Exp}} may be any term including variables previously marked with a \q{?}.

For example, the condition:
\begin{alltt}
X :: expression
\end{alltt}
means that the term bound to the meta-variable \q{X} should satisfy the validation conditions for the \q{expression} category.

\section{Evaluation of Validation Rules}
\label{validationEvaluation}

The validation process is driven by category. For example, the rule:
\begin{alltt}
# for ?C do ?B :: action :- C :: condition :& B :: action;
\end{alltt}
defines that a \q{for}\ldots\\q{do} is a legal \q{action} provided that the condition is a \q{condition} and the body is also an \q{action}.

The left hand side of the rule is a pattern matched against fragments of source program. The right hand side is a conjunction (in this case) of two sub-validation goals. The validation goal
\begin{alltt}
C :: condition
\end{alltt}
is satisfied by attempting the rules for validating \q{condition}s against the fragment bound to \q{C}.

Each validation rule left hand side has a \emph{specificity} -- similar to the specificity of macro rules (see Section~\vref{macroSpecificity}) -- which is effectively a measure of how specific the validation rule is.

Validation rules are attempted in a most specific first order; i.e., the most specific validation rule that may match a program fragment is the one used.


\section{Standard Syntactic Categories}
\label{standardCategories}
The profile developer is free to define new validation categories. However, some categories are standard and have a standard interpretation. These are listed below:
\begin{description}
\item[statement] A statement in the language. Includes function declarations, type declarations and groups of statements.
\item[action] An action that may be performed.
\item[expression] An expression that has a value.
\item[pattern] A pattern that is expected to matched against a value. Patterns may result in variables being bound. 
\item[identifier] An explicit identifier.
\item[symbol] An explicit symbol.
\item[character] An explicit character literal.
\item[string] An explicit string literal.
\item[number] A numeric literal. This includes floating point literals and integer literals.
\item[float] A floating point literal.
\item[fixed] A fixed point literal.
\item[integer] An integer literal.
\item[natural] A natural number literal.
\end{description}
Other categories may be introduced in user-defined validation rules (see Section~\vref{ValidationRule}).

In addition to defining new validation categories, it is quite possible -- even normal -- for profile developers to define \emph{new} rules for existing categories. The most common of which is probable the \q{expression} category.

\begin{aside}
Any non-standard syntactic category must be \emph{erasable} by suitable macro expansion rules. The categories in this list are the ones that the \Sr compiler understands; the macro rules are used to rewrite new syntactic categories into core categories.
\end{aside}



