%!TEX root = reference.tex
\chapter{Computation Expressions}
\label{computation}
Computation expressions are a special form of expression notation that permits computations to be performed in an augmented fashion. One standard example is the \q{task} expression -- see Chapter~\vref{concurrent} -- where the computations identified may be performed in parallel or asynchronously.

The core concepts behind \ntRef{ComputationExpression}s are captured in three contracts -- the  \q{computation} contract (see Program~\vref{computationContractProg}), the \q{execution} contract (see Program~\vref{executionContractProg}), and the \q{injection} contract (see Program~\vref{injectionContractProg}). 

There is a standard transformation of \ntRef{ComputationExpression}s into uses of these contracts. An expression may be \emph{encapsulated} as a computation, \ntRef{ComputationExpression}s may be \emph{combined} and they may be \emph{performed} in order to access the value computed.

The `augmentation' of a \ntRef{ComputationExpression} depends on the mode of the expression -- its monad type. For example, the \q{task} expression allows computations to be interleaved and executed in parallel on a suitable processor. \q{task} expressions and general concurrency are covered in detail in Chapter~\vref{concurrent}.

\begin{aside}
The \ntRef{ComputationExpression} and the \q{computation} contract have an analogous relationship as Haskell's Monad class and it's \q{do} notation. However, the \q{computation} contract is not identical to the Monad class.
\end{aside}

\section{The \q{computation} contract}
\label{computationContract}
The \q{computation} contract defines two key concepts: the `encapsulation' of an expression as a computation that leads to the value of that expression; and the `combination' of two computations.


\begin{program}
\begin{lstlisting}
contract (computation) over m determines e is {
  _encapsulate has type for all t such that (t)=>m of t
  _combine has type for all s,t such that
      (m of s,(s)=>m of t)=>m of t
  _abort has type for all t such that (e)=>m of t
  _handle has type for all t such that
      (m of t,(e)=>m of t)=>m of t
  _delay has type for all t such that (()=>m of t)=>m of t
  _delay(F) default is _combine(_encapsulate(()),(_) => F())
}
\end{lstlisting}
\caption{The Standard \q{computation} Contract\label{computationContractProg}}
\end{program}

\begin{aside}
The name of the contract -- \q{(computation)} -- is parenthesized in Program~\vref{computationContractProg} and in other references to the contract. This is required because \q{computation} is the operator used to signal a \ntRef{ComputationExpression}.
\end{aside}

\begin{aside}
The \q{\_encapsulate} function corresponds to the Monad \q{return}; and the \q{\_combine} function corresponds to Monad \q{bind}. \q{\_abort} corresponds to \q{fail}. However, the \q{\_handle} and \q{\_delay} functions do \emph{not} typically appear in Monads.
\end{aside}

The higher-kinded type variable \q{c} mentioned in the \q{computation} contract denotes the Monad of the \ntRef{ComputationExpression}. The computation type involving \q{c} has a single argument -- which is used to denote the value associated with the \ntRef{ComputationExpression}s. For example, \q{task of integer} is the type of a \q{task} expression whose value is an \q{integer}.

\subsection{\q{\_encapsulate} -- encapsulate a computation value}
\label{encapsulateFunction}
\index{computation contract@\q{computation} contract!_encapsulate@\q{\_encapsulate}}
\q{\_encapsulate} is part of the standard \q{computation} contract.
\begin{alltt}
\_encapsulate has type for all m,t,e such that (t)=>m of t
                       where (computation) over m determines e
\end{alltt}
The \q{\_encapsulate} function is used to encapsulate a value into a computation that has the value as its value. I.e., the \q{\_encapsulate} function is at the core of providing the additional indirection between values and computations returning those values.

\begin{aside}
If a computation has no value associated with it then \q{\_encapsulate} should be invoked with the empty tuple -- \q{()}.
\end{aside}

\subsection{\q{\_combine} -- combine two computation values}
\label{combineFunction}
\index{computation contract@\q{computation} contract!_combine@\q{\_combine}}
\q{\_combine} is part of the standard \q{computation} contract.
\begin{alltt}
\_combine has type for all m,s,t,e such that (m of s, (s)=>m of t)=>m of t
                   where (computation) over m determines e
\end{alltt}
The \q{\_combine} function constructs a new computation value by applying a transforming function to an existing computation value. Typically, the transforming function represents the `next step' in the computation.

\subsection{\q{\_abort} -- abort a computation}
\label{abortFunction}
\index{computation contract@\q{computation} contract!_abort@\q{\_abort}}
\q{\_abort} is part of the standard \q{computation} contract.
\begin{alltt}
\_abort has type for all m,t,e such that (e)=>m of t
                 where (computation) over m determines e
\end{alltt}

The \q{\_abort} function is used to represent a failed computation. \q{\_abort} takes a single argument -- of type \q{e} -- and returns a computation value.

\begin{aside}
Normally, the \q{\_abort} implementation will wrap the \q{e} value in a way that a subsequent \q{\_handle} or \q{\_perform} can leverage.
\end{aside}

When an \q{\_abort}ed computation is \q{\_perform}ed; the abort handler function will be  invoked with the value passed in to \q{\_abort}.

\subsection{\q{\_handle} -- handle an aborted computation}
\label{handleFunction}
\index{computation contract@\q{computation} contract!_handle@\q{\_handle}}
\q{\_handle} is part of the standard \q{computation} contract.
\begin{alltt}
\_handle has type for all m,t,e such that 
                  (m of t, (e)=>m of t) => m of t
                  where (computation) over m determines e
\end{alltt}

The \q{\_handle} function is used to potentially recover from a failed computation -- while continuing the computation. If the first argument to \q{\_handle} represents an aborted computation, then the second argument -- a handler function -- is invoked with the exception. It is the responsibility of this handler function to either recover from the exception or to propagate the exception.

\subsection{\q{\_delay} -- construct a delayed computation}
\label{delayFunction}
\index{computation contract@\q{computation} contract!_delay@\q{\_delay}}
\q{\_delay} is part of the standard \q{computation} contract.
\begin{alltt}
\_delay has type for all m,t,e such that (()=>m of t)=>m of t
                 where (computation) over m determines e
\end{alltt}
The \q{\_delay} function constructs a new `delayed' computation value. It is used in the construction of \ntRef{ComputationExpression}s -- at the top level -- to ensure that \ntRef{ComputationExpression}s are evaluated at the appropriate time.

The \q{\_delay} function has a default implementation -- which may be used in case that a particular implementation of \q{computation} does not require a specific implementation. The default implementation is:
\begin{alltt}
_delay(F) default is _combine(_encapsulate(()),(_) => F())
\end{alltt}

\section{The \q{execution} Contract}
\label{execution}

The \q{execution} contract has a single function defined in it -- encapsulating the concept of performing a computation.

\begin{program}
\begin{alltt}
contract execution over m is \{
  _perform has type for all t,e such that (m of t) => t;
\}
\end{alltt}
\caption{The Standard \q{execution} Contract\label{executionContractProg}}
\end{program}

\subsection{\q{\_perform} -- dereference a computation value}
\label{performFunction}
\index{execution contract@\q{execution} contract!_perform@\q{\_perform}}
\q{\_perform} is part of the standard \q{execution} contract.
\begin{alltt}
\_perform has type for all m,t,e such that (m of t)=>t
                   where (computation) over m
\end{alltt}
The \q{\_perform} function is used to `extract' the value of a computation. As such it is the natural inverse to the \q{\_encapsulate} function.

If the computation fails, then, typically, an exception will be raised -- see Section~\vref{exceptionType}.

\begin{aside}
The standard monad does \emph{not} include the equivalent of a \q{\_perform}. One reason being that not all encapsulation functions have an inverse.
\end{aside}

\section{The \q{injection} Contract}
\label{injection}

The \q{injection} contract refers to the `injection' of one computation into another. This occurs most often when a \ntRef{ComputationExpression} contains a \q{perform} action. Such an action represents an `injection' of the inner performed monad into the outer monad.

\begin{program}
\begin{alltt}
contract injection over (m,n) is \{
  \_inject has type for all t such that (m of t)=>n of t;
\}
\end{alltt}
\caption{The Standard \q{injection} Contract\label{injectionContractProg}}
\end{program}

The \q{injection} contract is a multi-type contract. I.e., implementations of the \q{injection} contract necessarily mention two types: the source Monad and the destination Monad. 

\subsection{\q{\_inject} -- inject one computation into another}
\label{injectFunction}
\index{injection contract@\q{injection} contract!_inject@\q{\_inject}}
\q{\_inject} is part of the standard \q{injection} contract.
\begin{alltt}
\_inject has type for all m,n,t such that (m of t)=>n of t
                  where injection over (m,n)
\end{alltt}
The \q{\_inject} function is used to migrate a computation from one monad to another. 

There are two primary requirements for the \q{\_inject} function: a normal computation must be migrated as a normal computation in the target monad; and an aborted computation must be represented as an aborted computation. 

In addition to implementing injection in a pairwise manner between monads, it is advisable to implement null-ary `self injection' -- i.e., to and from the same monad.

\subsection{Monadic Laws}
\label{monadicAxioms}
Additionally to the type signatures of the functions defined in the \q{computation} contract, \ntRef{ComputationExpression}s depend on some additional properties of any implementations of the contract.
\begin{aside}
These laws are assumed -- they cannot be verified by the compiler. In particular, if the \q{computation} contract is implemented for a user-defined type, then the \q{implementation} must respect the laws identified here.
\end{aside}

The first law relates the \q{\_encapsulate} and the \q{\_combine} functions. Specifically, if we combine an \q{\_encapsulate} with a \q{\_combine} the value is the same as applying the encapsulated value to the combining function:
\[\begin{array}{rcl}
\q{\_combine(\_encapsulate(X),F)}&=&\q{F(X)}
\end{array}\]

The second law is the complement, combining with encapsulation itself leaves the result alone:
\[\begin{array}{rcl}
\q{\_combine(X,\_encapsulate)}&=&\q{X}
\end{array}\]

The third law expresses the associativity of \q{\_combine}:
\[\begin{array}{rcl}
\q{\_combine(X, (U) => combine(F(U),G))}&=&\q{\_combine(\_combine(X,F),G)}
\end{array}\]

The abort law expresses the meaning of \q{\_abort}:
\[\begin{array}{rcl}
\q{\_combine(\_abort(E),\_)}&=&\q{\_abort(E)}
\end{array}\]
I.e., once a computation is aborted, then it effectively stops -- unless it is handled.

The handle law expresses how aborted computations may be recovered from:
\[\begin{array}{rcl}
\q{\_handle(\_abort(E),F)}&=&\q{F(E)}\\
\q{\_handle(\_encapsulate(X),\_)}&=&\q{\_encapsulate(X)}
\end{array}
\]
`Handling' an encapsulated computation -- i.e., a normal non-aborted computation -- has no effect.

\section{The \q{action} Monad}
\label{actionMonad}

The \q{action} type may be used to represent normal actions as \ntRef{ComputationExpression}s. The \q{action} type is defined in Program~\vref{actionTypeProg}.

\begin{program}
\begin{alltt}
type action of t is 
     _delayed(()=>action of t)
  or _aborted(exception)
  or _done(t);
\end{alltt}
\caption{The \q{action} Contract\label{actionTypeProg}}
\end{program}

The different constructors in the \q{action} type are intended to represent the three `phases' of an action computation: \q{\_done} denotes a completed computation, \q{\_delayed} represents a suspended computation and \q{\_aborted} denotes a failed computation.

The standard implementations of the \q{computation}, \q{execution} and the nullary implementation of the \q{injection} contract are shown in Program~\vref{actionImplementationProgram}.

\begin{program}
\begin{lstlisting}
implementation (computation) over action determines exception is {
  fun _encapsulate(V) is _done(V);
  fun _abort(E) is _aborted(E);
  fun _handle(A,H) is runCombo(A,_encapsulate,H)
  fun _combine(A,C) is _delayed(()=>runCombo(A,C,_abort))
  fun _delay(F) is _delayed(F);
}
implementation execution over action determines exception is {
  _perform(A,H) is runCombo(A,id,H)
};
implementation injection over (action,action) is {
  _inject(C) is C;
}
private
fun runCombo(_delayed(D),C,H) is runCombo(D(),C,H)
 |  runCombo(_done(X),C,_) is C(X)
 |  runCombo(_aborted(E),_,H) is H(E)
\end{lstlisting}
\caption{Implementation of Standard \q{execution} Contracts for the \q{action} Monad\label{actionImplementationProgram}}
\end{program}

\section{Computation Expressions}
\label{computationExpression}

A \ntRef{ComputationExpression} is a special syntax for writing expressions involving the various computation contracts. The compiler will automatically translate \ntRef{ComputationExpression}s into appropriate combinations of the functions in the \q{computation}, \q{execution} and \q{injection} contracts.

A \ntRef{ComputationExpression} consists of an \ntRef{ActionBlock}; i.e., a sequence of \ntRef{Action}s preceded by the \q{computation} keyword and the name of a generic unary type -- as defined in Figure~\vref{computationExpressionFig}.

\begin{figure}[H]
\begin{eqnarray*}
\emph{Expression}&\arrowplus&\ntRef{ComputationExpression}\\
\ntDef{ComputationExpression}&\arrow&\ntRef{Identifier}\ \q{computation}\ \q{\{}\ \ntRef{Action}\,\sequence{;}\,\ntRef{Action}\,\q{\}}\end{eqnarray*}
\caption{Computation Expression}
\label{computationExpressionFig}
\end{figure}

The type identified in the \ntRef{ComputationExpression} must implement the \q{computation} contract. For example, the \q{maybe} type:
\begin{alltt}
type maybe of \%t is possible(\%t) or impossible(exception)
\end{alltt}
might have the implementation defined in Program~\vref{maybeProgram} for the \q{computation} contract.

\begin{program}
\begin{lstlisting}
implementation (computation) over maybe determines exception is {
  fun _encapsulate(X) is possible(X);
  fun _combine(possible(S),F) is F(S);
   |  _combine(impossible(R),_) is impossible(R);
  fun _abort(Reason) is impossible(Reason);
  fun _handle(M matching possible(_),_) is M;
   |  _handle(impossible(E),F) is F(E);
}
\end{lstlisting}
\caption{Implementing the  \q{computation} contract for \q{maybe}\label{maybeProgram}}
\end{program}

Given such a definition, we can construct \q{maybe} \ntRef{ComputationExpression}s, such as in the function \q{find} in:
\begin{alltt}
fun find(K,L) is maybe computation \{
  for (KK,V) in L do\{
    if K=KK then
      valis V;
  \};
  raise "not found";
\};
\end{alltt}

Note that the \q{find} function does \emph{not} directly look for a value in a sequence. The value of a call to \q{find} is a computation that, when evaluated, will return the result of looking for a value.

\subsection{Accessing the value of a computation expression}
Where the \ntRef{ComputationExpression} notation is used to construct a computation; the \q{valof} form is used to access the value denoted.

There are two variations of \q{valof} expressions, outlined in Figure~\vref{valofFig}.

\begin{figure}[hbtp]
\begin{eqnarray*}
\emph{Expression}&\arrowplus&\ntRef{ValofComputation}\\
\ntDef{ValofComputation}&\arrow&\q{valof}\ \ntRef{Expression}\\
&\choice&\q{valof}\ \ntRef{Expression}\ \q{on abort}\ \ntRef{CaseActionBody}
\end{eqnarray*}
\caption{Valof computation expression}
\label{valofFig}
\end{figure}

The first form simply accesses the value associated with the computation -- and assumes that it was successful. For example, given a list:
\begin{alltt}
M is list of [(1,"alpha"), (2,"beta"), (3,"gamma"), (4,"delta")];
\end{alltt}
then the expression:
\begin{alltt}
valof find(2,MM)
\end{alltt}
will have value
\begin{alltt}
"beta"
\end{alltt}

The second form uses an \q{on abort} handler to cope with reported failure in the \ntRef{ComputationExpression}. For example, the expression:
\begin{alltt}
valof ff(5,MM) on abort \{ exception(_,E cast string,_) do valis E \}
\end{alltt}
will have value the string \q{"not found"}.

\subsection{Performing a computation}
\label{performComputation}
The \ntRef{PerformComputation} is the analog of \ntRef{ValofComputation} where the computation is an action that does not have a return value.

\begin{figure}[hbtp]
\begin{eqnarray*}
\emph{Action}&\arrowplus&\ntRef{PerformComputation}\\
\ntDef{PerformComputation}&\arrow&\q{perform}\ \ntRef{Expression}\\
&\choice&\q{perform}\ \ntRef{Expression}\ \q{on abort}\ \ntRef{ActionCaseBody}
\end{eqnarray*}
\caption{Perform Computation Action}
\label{performFig}
\end{figure}

The \q{perform} action is used when an action -- typically in a sequence of actions -- is the performance of a \ntRef{ComputationExpression}.

\begin{aside}
The type of computation being \q{perform}ed \emph{does not} have to be the same as the performing computation. For example, it is permissible to mix \q{task} computations with \q{maybe} computations:
\begin{alltt}
TT is task\{
  perform ff(5,MM)
\}
\end{alltt}
Note that, as with all actions, any value returned by the performed computation is discarded.
\end{aside}

\begin{aside}
\q{perform} within a \ntRef{ComputationExpression} denotes a use of the \q{\_inject} function. I.e., the \q{perform}:
\begin{alltt}
perform ff(5,MM)
    on abort \{ exception(_,E case string,_) do logMsg(info,E) \}
\end{alltt}
is represented by the expression:
\begin{alltt}
\_inject(ff(5,MM), (E) => valof\{ logMsg(info,E); valis () \})
\end{alltt}
The normal overloading rules will ensure that the appropriate implementation of injection between monads is invoked.
\end{aside}

\subsection{Handling Failure}
\label{failAction}

The \ntRef{OnAbort} action is used to handle a failed (i.e., \q{\_abort}ed) computation -- while continuing the \ntRef{ComputationExpression} itself.

\begin{figure}[hbtp]
\begin{eqnarray*}
\emph{Action}&\arrowplus&\ntRef{OnAbort}\\
\ntDef{OnAbort}&\arrow&\q{try}\ \ntRef{Action}\ \q{on abort}\ \ntRef{ActionCaseBody}
\end{eqnarray*}
\caption{Abort Handler Action}
\label{failHandleFig}
\end{figure}

The \q{on abort} action is used to recover from a failed computation. The \ntRef{ActionCaseBody} is a rule that matches the failure and performs appropriate recovery action. For example, the action in:
\begin{alltt}
task\{
  try P(X) on abort \{ E do logMsg(info,"exceptional \$E") \}
\}
\end{alltt}
calls the procedure \q{P}; but if that results in an abort, then the abort handler is entered with the variable \q{E} being matched against the exception. 

The type of the exception variable is the standard type \q{exception}.

It is equivalent to a call of the contract function \q{\_handle}. I.e., the above action is equivalent to:
\begin{alltt}
\_handle(P(X),
         (E) => valof \{ logMsg(info,"exceptional \$E"); valis () \})
\end{alltt}

\subsection{\q{action} Expressions}
\label{actionExpression}
A basic variant of the \ntRef{ComputationExpression} is the \q{action} expression. \q{action} expressions take the form:
\begin{alltt}
action\{ \ntRef{Action}\sequence{;}\ntRef{Action} \}
\end{alltt}
which is shorthand for:
\begin{alltt}
action computation \{ \ntRef{Action}\sequence{;}\ntRef{Action} \}
\end{alltt}

\begin{aside}
There is a strong connection between \q{action} expressions and \ntRef{ValueExpression}s. In particular, we have the equivalence:
\[\begin{array}{rcl}
\q{valof\{} \ntRef{Action}\sequence{;}\ntRef{Action}\q{\}}&=&\q{valof action\{}\ntRef{Action}\sequence{;}\ntRef{Action}\,\q{\}}
\end{array}
\]
However, a crucial distinction between \q{action} expressions and \ntRef{ValueExpression}s is that the former may be manipulated and combined in addition to the value being determined.
\end{aside}

