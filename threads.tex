%!TEX root = reference.tex
\chapter{Concurrent Execution}
\label{threads}
\index{Concurrent execution}
\index{threads}

Concurrent execution involves the use of \q{spawn} (and the related parallel execution operator \q{//}) to initiate concurrent execution; together with a set of features intended to constrain that concurrent execution.

\begin{figure}[htbp]
\begin{eqnarray*}
\ntDef{Action}&\arrowplus&\ntRef{ParallelAction}\\
&\choice&\ntRef{SpawnAction}\\
&\choice&\ntRef{WaitAction}\\
&\choice&\ntRef{SyncAction}%\\
%&\choice&\ntRef{AtomicAction}
\end{eqnarray*}
\caption{Thread and Parallel Action}
\label{threadActionFig}
\end{figure}

\section{Spawning Concurrent Execution}
\label{threadSpawning}

There are three `levels' of operator relating to initiating and terminating concurrent execution: the parallel execution operator \q{//}, the \q{spawn} action/expression and the \q{threadStart} and \q{threadWaitfor} operators.

\subsection{Parallel Execution}
\label{parallelExecution}

The \q{//} operator is used to signal that two actions should proceed in parallel.
\begin{figure}[htbp]
\begin{eqnarray*}
\emph{ParallelAction}&\arrow&\ntRef{Action}\ \q{//}\ \ntRef{Action}
\end{eqnarray*}
\caption{parallel Execution Action}
\label{parallelExecutionFig}
\end{figure}

An action of the form:
\begin{alltt}
\emph{LeftAction}//\emph{RightAction}
\end{alltt}
executes both of \q{\emph{LeftAction}} and \q{\emph{RightAction}} -- in some indeterminate, possibly interleaved, order. The action itself does not terminate until both \q{\emph{LeftAction}} and \q{\emph{RightAction}} have terminated.

For example, to perform a \q{request} action (see Section~\vref{request}) to two different targets in parallel, one might use:
\begin{alltt}
\ldots;\{ request A to oneThing() \}//\{request B to otherThing()\};\ldots
\end{alltt}
It would be also possible to ask the \emph{same} target to perform operations in parallel but that raises issues relating to accessing shared resources. For one way of managing this, see Section~\vref{syncAction}. %and Section~\vref{transaction}.

\section{Thread Spawning}
\begin{aside}
The \q{spawn} action/expression is slightly more low-level than the \q{//} operator. The \q{fork} primitive is lower-level still; and gives the finest level of control of initiating a concurrent execution.
\end{aside}

\subsection{The \q{thread} Type}
\label{threadType}
\index{threads!type of}

The \q{thread} type is a standard type that is used to represent threads of execution. It does not have a standard constructor; and hence cannot be defined using normal type definition notation. However, it is generic; a type expression of the form:
\begin{alltt}
thread of integer
\end{alltt}
denotes the type of a thread which will ultimately yield an \q{integer} value when it completes. 

\begin{aside}
Not all threads yield a value; in which case the type expression is simply \q{thread}. (i.e., the type expressions \q{thread of \emph{type}} and \q{thread} refer to different kinds of threads -- in one case returning a value and in the other not.
\end{aside}

\subsection{Spawn Action}
\label{spawnAction}
\index{spawn action@\q{spawn} action}
\index{parallel execution}

A \q{spawn} action executes an \emph{Action} in parallel with the `main' action.
\begin{figure}[htbp]
\begin{eqnarray*}
\ntDef{SpawnAction}&\arrow&\q{spawn}\ \ntRef{Action}
\end{eqnarray*}
\caption{Spawning a thread}\label{spawnSyntaxFig}
\end{figure}
When executed, a \q{spawn} action completes `immediately' -- without waiting for the \q{spawn}ed action to complete. The \q{spawn}ed action itself is executed as a separate thread of activity.

\subsubsection{Variables in a \q{spawn}ed action}
\index{spawn action@\q{spawn} action!variables in a}
\index{variables in a \q{spawn} action}

Any variables that are \emph{free} in a \q{spawn}ed action are treated as read-only within the \q{spawn}ed action -- with the exception of \q{resource} variables (see Section~\vref{reassignableVars}). Thus an action of the form:
\begin{alltt}
\{
  var X := 1;
  \ldots
  spawn \{ X := X+1; \}
  \ldots
\}
\end{alltt}
is not legal since the variable \q{X} is not a \q{resource} variable, and hence it may not be reassigned to within the \q{spawn}ed action.

The value assigned to reassignable variables within a \q{spawn} action is `fixed' at the time that the \q{spawn} is performed.

However, in the fragment:
\begin{alltt}
\{
  resource var R := 1;
  \ldots
  spawn \{ R:=R+1; \}
  \ldots
\}
\end{alltt}
the variable \q{R} is a \q{resource} variable. Hence \q{R} may be reassigned to within the \q{spawn}ed action, and its value always reflects the last assignment to the variable.
\begin{aside}
Of course, this also represents a major risk to the safety of programs. The programmer should ensure that any reference or assignment to shared \q{resource} variables is in the context of an \q{atomic} action. Otherwise, there is a high risk of unexpected race conditions and corruption in the values of such variables.
\end{aside}

\subsubsection{Type Safety}

\begin{prooftree}
\AxiomC{\typesafe{E}{\emph{Action}}}
\UnaryInfC{\typesafe{E}{\q{spawn}\ \emph{Action}}}
\end{prooftree}

\subsection{Spawn Expression}
\label{spawnExpression}
The \q{spawn} expression -- likes its action counterpart (see Section~\vref{spawnAction}) -- is used to spawn off an expression evaluation to be evaluated concurrently with the invoking computation.

\begin{figure}[htbp]
\begin{eqnarray*}
\emph{Expression}&\arrowplus&\ntRef{SpawnExpression}\\
\ntDef{SpawnExpression}&\arrow&\q{spawn}\ \ntRef{Expression}
\end{eqnarray*}
\caption{Spawning an expression}\label{spawnExpressionSyntaxFig}
\end{figure}

\subsubsection{Type Safety}
The type of a \q{spawn} expression is \q{thread}. However, the \q{thread} type is generic, intended to denote the result of the \q{spawn}ed expression as it is re-captured by the corresponding \q{wait} expression.

\begin{prooftree}
\AxiomC{\typeprd{E}{Ex}{T}}
\UnaryInfC{\typeprd{E}{\q{spawn} Ex}{\q{thread of }T}}
\end{prooftree}

\subsection{Thread Wait Action}
\label{threadWaitAction}
\index{spawn action@\q{spawn} action!wait for termination}
The \q{waitfor} action blocks until an identified thread has terminated. If the \q{thread} has a value associated with it, then that value also becomes the value returned by \q{waitfor}.

\begin{figure}[htbp]
\begin{eqnarray*}
\emph{Action}&\arrowplus&\ntRef{WaitAction}\\
\ntDef{WaitAction}&\arrow&\q{waitfor}\ \ntRef{Expression}
\end{eqnarray*}
\caption{WaitFor action}\label{waitforActionSyntaxFig}
\end{figure}

\subsection{\q{wait} expression}
\label{waitExpression}
The \q{wait} expression takes a \q{thread} as an argument and has as its value the value returned by the \q{thread}. The \q{wait} expression does suspends execution until the \q{thread} has completed.

\begin{figure}[htbp]
\begin{eqnarray*}
\emph{Expression}&\arrowplus&\ntRef{WaitExpression}\\
\ntDef{WaitExpression}&\arrow&\q{waitfor}\ \ntRef{Expression}
\end{eqnarray*}
\caption{WaitFor expression}\label{waitforExpressionSyntaxFig}
\end{figure}

\subsubsection{Type Safety}
The type of a \q{spawn} expression is \q{thread}. The type of a \q{wait} expression is the type of the value returned by the \q{thread}.

\begin{prooftree}
\AxiomC{\typeprd{E}{E\sub{spawn}}{\q{thread of }T}}
\UnaryInfC{\typeprd{E}{\q{wait}\ E\sub{spawn}}{T}}
\end{prooftree}

\subsection{The \q{sync}hronized Action}
\label{syncAction}
\index{sync action@\q{sync} action}

The \q{sync} action is used to manage contention in accessing resources that are potentially shared across \q{spawn}ed actions. There are two forms of \q{sync} action -- a standard form and the guarded form. In both cases, \q{sync} revolves around access to a shared resource.

\begin{figure}[htbp]
\begin{eqnarray*}
\ntDef{SyncAction}&\arrow&\q{sync(}\ntRef{Expression}\q{)\{}\ntRef{SyncBody}\q{\}}\\
\ntDef{SyncBody}&\arrow&\ntRef{SyncGuard}\sequence{;}\ntRef{SyncGuard}\\
&\choice&\ntRef{Action}\\
\ntDef{SyncGuard}&\arrow&\q{when}\ \ntRef{Condition}\ \q{do}\ \ntRef{Action}
\end{eqnarray*}
\caption{Synchronized Action}\label{syncActionSyntaxFig}
\end{figure}

The \q{sync} action's resource \emph{Expression} may evaluate to any term value; however, we recommend that `something obvious' is used. For example, in the context of a set of actions revolving around a \ntRef{ThetaRecord} the \q{this} keyword identifies the \ntRef{ThetaRecord} itself.

Only one \q{sync} action may be executing at any given time on a given resource. If another \q{spawn}ed action is executing a \q{sync} action involving the same resource then this action is paused until that action completes.

If the guarded form is used, each \ntRef{SyncGuard} defines a condition that must be satisfied in order for access to the shared resource to be valid. These guards are evaluated in a left-to-right order: the first guard to be satisfied fires its corresponding action.

The process of acquiring exclusive access to a shared resource when using \ntRef{SyncGuard}s can be described:
\begin{enumerate}
\item A lock on the shared resource is acquired.
\item For each \ntRef{SyncGuard}, its guard \ntRef{Condition} is evaluated:
\begin{itemize}
\item  If the guard is satisfied, then the corresponding guarded \ntRef{Action} is performed. At the end of which the lock on the shared resource is released and the \ntRef{SyncAction} is completed.
\item If the guard is not satisfied, then the next \ntRef{SyncGuard} is considered.
\end{itemize}
\item If no remaining \ntRef{SyncGuard} exists; the lock on the shared resource is released, the \ntRef{SyncAction} is suspended, and a new attempt on a lock on the shared resource will be made after another thread has successfully acquired and released the lock. At which point, execution of the \ntRef{SyncAction} will be restarted.
\end{enumerate}

\begin{aside}
This is equivalent to the `monitor' exclusion pattern for controlling access to shared resources.
\end{aside}

\begin{aside}
The \q{sync} action is a fairly low-level mechanism. It can be difficult to use \q{sync} actions to achieve high performance. However, it can be used to build higher-level mechanisms such as semaphores and atomic transactions. For example, Program~\vref{semaphoreProg} shows how a semaphore can be implemented in terms of the \q{sync} action.
%
%We recommend using the \q{atomic} action (see Section~\vref{atomicAction}) as an alternative easier-to-use mechanism for most regular scenarios.
\end{aside}
\begin{program}
\begin{alltt}
semaphore(Count) is \{
  private var Lvl := Count;
    
  grab() do \{
    sync(this)\{
      when Lvl>0 do \{
        Lvl := Lvl-1;
      \}
    \};
  \};
    
  release() do \{
    sync(this)\{
      Lvl := Lvl+1;
    \}
  \}
\}
\end{alltt}
\caption{A Semaphore Generating Function\label{semaphoreProg}}
\end{program}

\subsubsection{Type Safety}
\begin{prooftree}
\AxiomC{\typeprd{E}{\emph{S}}{\emph{t}}}
\AxiomC{\typesafe{E}{\emph{Action}}}
\BinaryInfC{\typesafe{E}{\q{sync(}\emph{S}\q{)\{}\emph{Action}\q{\}}}}
\end{prooftree}

%\section{Transactional Isolation of Concurrent Threads}
%\label{transaction}



%\subsection{Atomic Actions}
%\label{atomicAction}
%\index{atomic actions}
%\index{transactions}
%
%An \q{atomic} action performs its argument action in a way that is \emph{atomic} with respect to other parallel activities -- such as those \q{spawn}ed off. In particular, other threads cannot see any changes to variables, or any speech actions that are performed, until the \q{atomic} action is completed.
%
%\begin{figure}[htbp]
%\begin{eqnarray*}
%\ntDef{AtomicAction}&\arrow&\q{atomic}\ \emph{Action}
%\end{eqnarray*}
%\caption{Atomic Action}\label{atomicSyntaxFig}
%\end{figure}
%
%\q{atomic} actions denote an equivalent of \emph{transactions}. They are intended to support \emph{isolation} of side-effects between concurrent activities; and hence enable better management of interactions.
%
%\subsubsection{Type Safety}
%
%\begin{prooftree}
%\AxiomC{\typesafe{E}{\emph{Action}}}
%\UnaryInfC{\typesafe{E}{\q{atomic}\ \emph{Action}}}
%\end{prooftree}
