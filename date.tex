%!TEX root = reference.tex
\chapter{Date and Time}
\label{dateAndTime}

\section{The \q{date} Type}
Date and time support revolves around the \q{date} built-in type. The type definition for \q{date} is straightforward:

\begin{alltt}
type date is date(\_long) or never
\end{alltt}

\begin{aside}
The \q{\_long} argument to the constructor is a so-called `raw value', not to be confused with the \q{long} built-in type (see Section~\vref{longType}). The \q{\_long} value is the number of milliseconds since Jan 1, 1970.

Under normal circumstances, programmers will never see the contents of the \q{date} constructor.
\end{aside}

The \q{never} enumerated symbol denotes a nonexistent date.

There are standard implementations of the \q{pPrint} (see Section~\vref{pPrintContract}) and \q{coercion} {see Section~\vref{typeCoercionExpression}) contracts for the \q{date} type. Thus, it is possible to parse a \q{string} as a \q{date} using an expression:
\begin{alltt}
S as date
\end{alltt}
In particular, \q{coercion} is implemented to support conversion between \q{date}$\leftrightarrow$\q{string} and \q{date}$\leftrightarrow$\q{long}.

\section{Date Functions}

\subsection{\q{now} -- Current Time}
\index{current time}
\label{nowFun}
Report the current time.
\begin{alltt}
now has type ()=>date;
\end{alltt}

Reports the current system time as a \q{date}.

\subsection{\q{today} -- Time at Midnight}

\begin{alltt}
today has type ()=>date;
\end{alltt}

Reports the time at midnight this morning as a \q{date}.

\subsection{\q{smallest} -- the earliest legal point in time}
\label{smallestDate}
\index{date@\q{date}!first date}
The \q{smallest} \q{date} is the first legal point in time. It corresponds to midnight Jan 1, 1970.
\begin{alltt}
smallest has type date
\end{alltt}
\q{smallest} is part of the \q{largeSmall} contract -- see Program~\vref{largeSmallProg}.

\subsection{\q{largest} -- the last legal point in time}
\label{largestDate}
\index{date@\q{date}!last date}

\begin{alltt}
largest has type date
\end{alltt}
\q{largest} is part of the \q{largeSmall} contract -- see Program~\vref{largeSmallProg}.

The \q{largest} \q{date} is the last legal \q{date} value. It corresponds to August 16, 292278994 11:12:55 PM PST.


\subsection{\q{\_format} -- Format a \q{date} as a \q{string}}
\label{formatDate}
\index{date!formatting}
\index{formatting a date string}
\index{convert to a date string}

\begin{alltt}
\_format has type (date,string)=>pP;
\end{alltt}

\begin{aside}
\q{\_format} is part of the \q{formatting} contract -- see Program~\vref{formatContractProg}.
\end{aside}

The \q{\_format} function computes a readable string representation of a \q{date} as a string displaying the date and/or time. The second argument is a format string that guides how to format the string.

The format string consists of letters, spaces and other characters; the letters control the representation of some aspect of the date, other non-letter characters are displayed as is in the result. Table~\vref{tableFormatTbl} contains the definitions of the available formatting characters.

\begin{table}[h]
\begin{center}
\caption{Date Formatting Control Letters}\label{tableFormatTbl}
\begin{tabular}{|llll|}
\hline
Letter&Date Component&Presentation&Examples\\
\hline
\tt G&Era designator&Text&\tt AD\\
\tt y&Year&Year&\tt 1999; 01\\
\tt M&Month in year&Month&\tt July; Jul; 07\\
\tt w&Week in year&Number&\tt 25\\
\tt W&Week in month&Number&\tt 2\\
\tt D&Day in year&Number&\tt 191\\
\tt d&Day in month&Number&\tt 2\\
\tt F&Day of week in month&Number&\tt 1\\
\tt E&Day in week&Text&\tt Tuesday; Tue\\
\tt a&AM/PM&Text&\tt PM\\
\tt H&Hour in day (0-23)&Number&\tt 0\\
\tt k&Hour in day (1-24)&Number&\tt 24\\
\tt h&Hour in day (1-12)&Number&\tt 11\\
\tt m&Minute in hour&Number&\tt 34\\
\tt s&Second in minute&Number&\tt 56\\
\tt S&Millisecond in second&Number&\tt 543\\
\tt z&General time zone&Text&\tt PDT; GMT-08:00\\
\tt Z&RFC 822 time zone&Text&\tt -0800\\
\hline
\end{tabular}
\end{center}
\end{table}

The \q{\_format} function may be invoked implicitly in a string \ntRef{Interpolation} expression. For example, the expression:
\begin{alltt}
"\$(now()):dd-MMM-yyyy hh:mm:ss;"
\end{alltt}
is equivalent to the expression:
\begin{alltt}
\_format(now(),"dd-MMM-yyyy hh:mm:ss")
\end{alltt}
\begin{aside}
This makes more of a difference when combined with other formatting and displaying, as in:
\begin{alltt}
logMsg(info,"Balance on \$(today()):dd-MMM-yy; is \euro\$Amnt:9,999.00;");
\end{alltt}
which will result in a line of the form:
\begin{alltt}
Balance on 24-Mar-13 is \euro{}23.56
\end{alltt}
being displayed.
\end{aside}

\subsubsection{Repeated Date Formatting Control Characters}
The repeated pattern control characters sometimes change their meaning when repeated:
\begin{description}
\item[Text] If the control character is repeated 4 or more times then the \emph{long} form of display is used when appropriate. Otherwise a short form is used.
\item[Year] If the control character is repeated 2 times -- i.e., if \q{yy} is used as the year format -- then the year is truncated to two digits. Otherwise, the year is printed in full.
\item[Month] If the \q{M} control is repeated 3 or more times then the text name of the month is used; (full name for 4 or more repetitions, short name for 3 repetitions). Otherwise, a numeric number is displayed.
\item[Number] The numeric value is displayed, with zero padding to ensure at least as many digits as format control characters.
\end{description}

\subsection{\q{parse\_date} -- Parse a \q{string} as a \q{date}}
\label{parseDate}
\index{date!formatting}
\index{parse a date string}

\begin{alltt}
parse\_date has type (string,string)=>date;
\end{alltt}

The \q{parse\_date} function parses a \q{string} into a \q{date} value. The first argument is the \q{string} to parse, the second is a format control string used to guide the parsing of the date. The form of the format control string is the same as for \q{format\_date} (see Section~\vref{formatDate}).

If the string cannot be parsed as a valid date using the control string, then the value returned is \q{never}.


\subsection{\q{timeDiff} -- Compute difference between two \q{date}s}
\label{timeDiff}
\index{date!difference between two dates}

\begin{alltt}
timeDiff has type (date,date)=>long
\end{alltt}

The \q{timeDiff} function `subtracts' one \q{date} from another returning the difference as a number of milliseconds.

\subsection{\q{timeDelta} -- Apply a delta to a \q{date}}
\label{timeDelta}
\index{date!add delta to a date}

\begin{alltt}
timeDelta has type (date,long)=>date
\end{alltt}
The \q{timeDelta} function adds an increment to a \q{date} -- expressed as a number of milliseconds -- and returns a new \q{date}.
\begin{aside}
The increment may be negative; for example, to compute yesterday's \q{date}, the following expression suffices:
\begin{alltt}
timeDelta(today(),-86400000L)
\end{alltt}
\end{aside}