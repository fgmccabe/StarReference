%!TEX root = reference.tex
\chapter{JSON}
\index{JSON}\index{Using the JSON type}

The JSON Infoset type, or just \q{json} type, allows values to be represented in a way that is easily digestible by many web-based tools -- including browsers. The \q{json} type is semantically equivalent to the JSON structure defined in \cite{rfc4627}. However, the \q{json} type represents a statically typed representation of JSON values.

In addition to basic handling of JSON values, \Sr provides a form of path notation that allows \q{json} values to be probed and updated.

\section{The \q{json} Type}
\label{infosetType}
\index{json type@\q{json} type}

Program~\vref{infosetProg} defines the \q{json} type.
\begin{program}
\begin{lstlisting}
type json is 
      iFalse or
      iTrue or
      iNull or
      iColl(dictionary of (string,json)) or
      iSeq(list of json) or
      iText(string) or
      iNum(long) or
      iFlt(float);
\end{lstlisting}
\caption{The \q{json} Type\label{infosetProg}}
\end{program}

\begin{aside}
JSON values are not strongly typed in the sense that the value associated with the \q{Width} of the \q{Thumbnail} is a string even though one might expect widths to be integral. However, JSON values are checked to be consistent with the \q{json} type -- like all other values.
\end{aside}

For example, the JSON value:
\begin{alltt}
\{
  "Image": \{
    "Width":  800,
    "Height": 600,
    "Title":  "View from 15th Floor",
    "Thumbnail": \{
      "Url":    "http://www.example.com/image/481989943",
      "Height": 125,
      "Width":  "100"
    \},
    "IDs": [116, 943, 234, 38793]
  \}
\}
\end{alltt}
can be represented using the \q{json} value shown in Figure~\vref{infosetEx}.

\begin{figure}[hbt]
\begin{alltt}
iColl(dictionary of [
  "Image" -> iCol(dictionary of [
    "Width" -> iNum(800l),
    "Height" -> iNum(600l),
    "Title" -> iText("View from 15th Floor"),
    "Thumbnail" -> iColl(dictionary of [
      "Url" -> iText("http://www.example.com/image/481989943"),
      "Height" -> iNum(125l),
      "Width" -> iText("100")
    ]),
    "IDs" -> iSeq(list of [ 
      iNum(116l), iNum(943l), iNum(234l), iNum(38793)
    ])
  ])])
\end{alltt}
\caption{An Example \q{json} Value}
\label{infosetEx}
\end{figure}

\begin{aside}
The JSON standard specification is mute on the topic of numeric precision. We choose to represent integers as \q{long} values and floating point values as \q{float} (which is equivalent to \q{double} precision arithmetic).
\end{aside}

\section{Infoset paths}
\label{infoPath}
\index{path access to json@path access to \q{json}}

Infoset values are typically deeply nested structures involving both accessing dictionary-like collections and arrays. In order to make working with \q{json} values simpler we introduce the concept of an json path -- an \q{infoPath}.

An \q{infoPath} is a list of path elements -- each of which represents either an index into a sequence of \q{json} elements or the name of a member of a collection of elements. This is captured in the definition of the \q{infoPathKey}, as defined in Program~\vref{infoPathProg}.
\begin{program}
\begin{alltt}
type infoPathKey is kString(string) or kInt(integer);

type infoPath is alias of list of infoPathKey;
\end{alltt}
\caption{The \q{infoPathKey} and \q{infoPath} Types\label{infoPathProg}}
\end{program}

For example, the path expression that denotes the url of the thumbnail in Figure~\vref{infosetEx} is:
\begin{alltt}
list of [kString("Image"), kString("Thumbnail"), kString("Url")]
\end{alltt}
and the path that denotes the first id from the \q{IDs} sequence is:
\begin{alltt}
list of [kString("Image"), kString("IDs"), kInt(0)]
\end{alltt}

Infoset paths are used in several of the functions that are defined on \q{json} values.

\section{Standard Functions on Infoset Values}
\index{json@\q{json}!standard functions}
Several contracts are implemented for \q{json} values; including \q{indexable}, \q{iterable}, \q{indexed\_iterable}, \q{pPrint} and \q{coercion}.

\subsection{\q{\_index} access to \q{json}}
\index{json@\q{json}!standard functions!_index@\q{\_index}}
\index{_index@\q{\_index}}

The \q{\_index} function applies an \q{infoPath} to an \q{json} to obtain a portion of the \q{json} value. It's type is:
\begin{alltt}
\_index has type (infoPath,json)=>json
\end{alltt}
\q{\_index} is part of the \q{indexable} contract -- see Section~\vref{indexableContract}.

For example, the value of the first element of the value shown in Figure~\vref{infosetEx} is gotten with the expression (assuming that the value is bound to the variable \q{I}:
\begin{alltt}
I[list of [kString("Image"), kString("IDs"), kInt(0)]]
\end{alltt}
This has value
\begin{alltt}
iNum(116L)
\end{alltt}

\begin{aside}
The above expression is a synonym of
\begin{alltt}
\_index(I,list of [kString("Image"), kString("IDs"), kInt(0)])
\end{alltt}
\end{aside}

\subsection{\q{\_set\_indexed} -- Set a Value in an \q{json}}
\index{json@\q{json}!standard functions!_set_indexed@\q{\_set\_indexed}}
\index{_set_indexed@\q{\_set\_indexed}}

The \q{\_set\_indexed} function updates a value in an \q{json} -- depending on a path -- and returns the updated \q{json}.
The type of \q{\_set\_indexed} is given by:
\begin{spec}
\_set\_indexed has type (json,info path,json)=>json
\end{spec}
\q{\_set\_indexed} is part of the \q{indexable} contract.

\begin{aside}
This function does not update the original; it returns a new value.
\end{aside}

To use this function to change the title of the value in Figure~\vref{infosetEx} (again assuming that it is bound to an updateable variable \q{I}) one might use the action:
\begin{alltt}
I[list of [kString("Image"), kString("Title")]] := iText("A Better One")
\end{alltt}
which is a synonym for the action:
\begin{alltt}
I := I[list of [kString("Image"), kString("Title")]->iText("A Better One")]
\end{alltt}
which, in turn, is a synonym for:
\begin{alltt}
I := \_set_indexed(I,list of [kString("Image"),kString("Title")],
                  iText("A Better One"))
\end{alltt}

\subsection{\q{\_delete\_indexed} -- Remove a Value from an \q{json}}
\index{json@\q{json}!standard functions!_delete_indexed@\q{\_delete\_indexed}}
\index{_delete_indexed@\q{\_delete\_indexed}}

The \q{\_delete\_indexed} function removes a value in an \q{json} -- depending on a path -- and returns the modified \q{json}.
\begin{aside}
This function does not update the original; it returns a new value.
\end{aside}
The type of \q{\_delete\_indexed} is given by:
\begin{alltt}
\_delete\_indexed has type (json,info path)=>json
\end{alltt}
\q{\_delete\_indexed} is part of the \q{indexable} contract.

To use this function to remove the last ID from \q{IDs} in Figure~\vref{infosetEx} one might use the action:
\begin{alltt}
remove I[list of [kString("Image"), kString("IDs"), kInt(3)]]
\end{alltt}
which is a synonym for the action:
\begin{alltt}
I := I[without list of [kString("Image"),kString("IDs"), kInt(3)]]
\end{alltt}

\subsection{\q{\_index\_member} -- Test a Path in an \q{json}}
\index{json@\q{json}!standard functions!_index_member@\q{\_index\_member}}
\index{_index_member@\q{\_index\_member}}

The \q{\_index\_member} pattern succeeds if there is a designated element of the \q{json} and matches it against a pattern.

The type of \q{\_index\_member} is given by:
\begin{alltt}
\_index\_member has type (json)<=(json,infopath)
\end{alltt}
\q{\_index\_member} is part of the \q{indexable} contract.

The \q{\_index\_member} pattern is typically used in query conditions; such as:
\begin{alltt}
if I[list of [kString("Image")]] matches L then
\end{alltt}
This is equivalent to the condition:
\begin{alltt}
if (I,list of [kString("Image")]) matches \_index\_member then
\end{alltt}

\subsection{\q{\_iterate} -- Over an \q{json}}
\index{json@\q{json}!standard functions!_iterate@\q{\_iterate}}
\index{_iterate@\q{\_iterate}}

The \q{\_iterate} function is used when iterating over the elements of an \q{json}. 

The type of \q{\_iterate} is given by:
\begin{alltt}
\_iterate has type for all s such that
  (json,(json,IterState of s)=>IterState of s,
   IterState of s) => IterState of s
\end{alltt}
The \q{\_iterate} function is part of the \q{iterable} contract -- see Section~\vref{iterableContract}.

The \q{json} variant of the \q{\_iterate} function calls the `client function' for all of the `leaf' elements of an \q{json} value. For example, in the condition:
\begin{alltt}
X in I
\end{alltt}
where \q{I} is the \q{json} value shown in Figure~\vref{infosetEx}, then the client function will be called successively on the \q{json} values:
\begin{alltt}
iNum(800l)
iNum(600l)
iText("View from 15th Floor")
iText("http://www.example.com/image/481989943")
iNum(125l)
iText("100")
iNum(116l)
iNum(943l)
iNum(234l)
iNum(38793)
\end{alltt}
The query:
\begin{alltt}
list of \{ all X where iText(X) in I \}
\end{alltt}
will have value:
\begin{alltt}
list of [ 
  "View from 15th Floor",
  "http://www.example.com/image/481989943",
  "100"
]
\end{alltt}

\subsection{\q{\_indexed\_iterate} -- Over an \q{json}}
\index{json@\q{json}!standard functions!_indexed_iterate@\q{\_indexed\_iterate}}
\index{_indexed_iterate@\q{\_indexed\_iterate}}

The \q{\_indexed\_iterate} function is used when iterating over the elements of an \q{json}. A key difference between this and \q{\_iterate} is that \q{\_indexed\_iterate} involves the paths to each of the leaf elements of the JSON value.

The type of \q{\_indexed\_iterate} is given by:
\begin{alltt}
\_indexed\_iterate has type for all s such that
  (json,(infoPath,json,IterState of s)=>IterState of s,
   IterState of s) => IterState of s
\end{alltt}
The \q{\_indexed\_iterate} function is part of the \q{indexed\_iterable} contract -- see Section~\vref{iterableContract}.

The \q{\_indexed\_iterate} function calls the `client function' for all of the `leaf' elements of an \q{json} value; providing an \q{infoPath} expression for each leaf element processed.

The \q{\_indexed\_iterate} function is typically used in conditions of the form:
\begin{alltt}
K -> V in I
\end{alltt}
where \q{K} is a pattern that matches the key (\q{infoPath}), \q{V} is a pattern that matches the (leaf) value, and \q{I} is the \q{json} being queried.

For example, in the condition:
\begin{alltt}
K->V in I
\end{alltt}
where \q{I} is the \q{json} value shown in Figure~\vref{infosetEx}, then the client function will be called successively on the \q{infoPath->json} values:
\begin{alltt}
list of [kString("Image"), kString("Width")] -> iNum(800l)
list of [kString("Image"), kString("Height")] -> iNum(600l)
list of [kString("Image"), kString("Title")] ->
       iText("View from 15th Floor")
list of [kString("Image"), kString("Thumbnail"), kString("Url")] -> 
       iText("http://www.example.com/image/481989943")
list of [kString("Image"), kString("Thumbnail"), kString("Height")] -> 
       iNum(125l)
list of [kString("Image"), kString("Thumbnail"), kString("Width")] -> 
       iText("100")
list of [kString("Image"), kString("IDs"), kInt(0)] -> iNum(116l)
list of [kString("Image"), kString("IDs"), kInt(1)] -> iNum(943l)
list of [kString("Image"), kString("IDs"), kInt(2)] -> iNum(234l)
list of [kString("Image"), kString("IDs"), kInt(3)] -> iNum(38793)
\end{alltt}
The \q{\_indexed\_iterate} function is therefore useful when you want to both process all the leaves in an \q{json} but also to know where they are.


\section{Parsing and Displaying}
The standard contract for displaying values -- \q{pPrint} -- is implemented for the \q{json} type. In addition, a \q{string} value may be parsed as a \q{json} by using the coercion expression:
\begin{alltt}
"\{"Id" : 34 \} as json
\end{alltt}
has value:
\begin{alltt}
iColl(dictionary of \{ "Id" -> iNum(34L) \})
\end{alltt}

\subsection{\q{ppDisp} -- Display a \q{json} Value}
The \q{pPrint} contract is implemented for \q{json} values. The type of \q{ppDisp} is given by:
\begin{alltt}
ppDisp has type (json)=>pP
\end{alltt}
The \q{pPrint} contract is described in Section~\vref{pPrintContract}. This implementation means that when a \q{json} value is displayed, it is shown in legal JSON syntax.

