%!TEX root = reference.tex
\chapter{Actions}
\label{actions}
\index{action}

An action is the performance of an operation in a particular context. 

\begin{figure}[htbp]
\begin{eqnarray*}
\ntDef{Action}&\arrow&\ntRef{NullAction}\ \choice\ \ntRef{ActionBlock}\\
\ntDef{Action}&\choice&\ntRef{LocalVariable}\ \choice\ \ntRef{TypeAnnotation}\\
&\choice&\ntRef{Assignment}\ \choice\ \ntRef{InvokeAction}\ \choice\ \ntRef{IgnoreAction}\\
&\choice&\ntRef{ForLoop}\ \choice\ \ntRef{WhileLoop}\ \choice\ \ntRef{ConditionAction}\\
&\choice&\ntRef{CaseAction}\ \choice\ \ntRef{LetAction}\\
&\choice&\ntRef{ValisAction}\ \choice\ \ntRef{AssertAction}\\
&\choice&\ntRef{RaiseAction}\ \choice\ \ntRef{TryAction}
\end{eqnarray*}
\caption{Action}
\label{actionFig}
\end{figure}

\subsubsection{Actions and Type Safety}
\label{actionTypeSafety}
The meaning of type safety is somewhat different for actions than for expressions and functions: by definition, actions do not denote values in the way that expressions do.

However, type safety still applies to actions. In particular, different actions have different \emph{type constraints} that must be satisfied; for example, an assignment action is \emph{type safe} if the type of the variable is consistent with the expression and if the variable is a re-assignable variable.

We use the meta-predicate \safeinf{} to indicate that a particular action is type safe. An assertion of the form:
\[\typesafe{E}{A}\]
means that the action $A$ is type-consistent given the environment $E$. In fact, this predicate is equivalent to a normal type derivation involving the \q{()} type:
\[\typesafe{E}{A}\ \iff\ \typeprd{E}{A}{\q{()}}\]

\section{Binding Actions}
Many actions are operators that change the state of one of more variables. However, some actions may bind variables -- that is, establish a new variable.

The most important binding actions are local variable definitions (Section~\vref{localVar}) and assignments (Section~\vref{assignment}).

\subsection{Local Variable Definition}
\label{localVar}
\index{action!local variable}
\index{variables in actions}

A local variable may be introduced within a block -- see Section~\vref{blockAction} -- using the same syntax as a variable declaration statement -- see Section~\vref{VariableDeclaration}.
\begin{aside}
It is named a local variable simply because it's scope is limited to the block of actions that contain the declaration.
\end{aside}
\begin{figure}[htbp]
\begin{eqnarray*}
\ntDef{LocalVariable}&\arrow&\q{var}\ \ntRef{Identifier}\ \q{:=}\ \ntRef{Expression}\\
&\choice&[\q{var}]\,\ntRef{Pattern}\ \q{is}\ \ntRef{Expression}
\end{eqnarray*}
\caption{Local Variable}
\label{localVariableFig}
\end{figure}

The scope of a local variable declaration is from the local declaration itself to the end of the containing \emph{ActionBlock}.

\begin{aside}
It is an error for a variable to be referenced within its own definition. Recursive definitions are not permitted as \ntRef{LocalVariable}s.
\end{aside}

A local variable declared using the \q{var}\ldots\q{:=} form is \emph{re-assignable}; whereas a variable declared using the \q{is} form is not. The type of a re-assignable variable is a \q{ref}erence type (see Section~\vref{referenceType}). For example, given the \ntRef{LocalVariable} declaration:
\begin{alltt}
X := 3
\end{alltt}
then the variable \q{X} has type \q{ref integer}.

If the left hand side of an \q{is} local variable definition is an identifier, or is an unlabeled tuple, then the \q{var} prefix is not required. However, it is good practice to use \q{var} in situations that may be confusing.

\begin{aside}
Note that the left hand side of an \q{is} definition is a \q{Pattern}, not simply an \q{Identifier}. One primary use for this form is to allow the `unpacking' of function results. For example, the function \q{ddivide} returns a pair of values: the quotient and the remainder result of dividing the first argument by the second:
\begin{alltt}
ddivide(X,Y) is (X/Y,X\%Y)
\end{alltt}
We can unpack the results of a call to \q{ddivide} using a \ntRef{TuplePattern} on the left hand side of the declaration:
\begin{alltt}
var (Q,R) is ddivide(34,3)
\end{alltt}
which would have the effect of binding \q{Q} to 11, and \q{R} to 1.
\begin{aside}
The reason that we get \q{integer} division with this call to \q{ddivide} is that the arguments to \q{ddivide} -- \q{34} and \q{3} -- are \q{integer}. The slightly different call:
\begin{alltt}
var (FQ,FR) is ddivide(34.0,3.0)
\end{alltt}
relies on \q{float}int point \q{arithmetic} and results in binding \q{FQ} to 11.333333, and \q{FR} to 1.0.
\end{aside}
\end{aside}

\begin{aside}
Local variables may be reassigned by an assignment action anywhere \emph{in the same} block as the variable declaration itself. For example, the following, somewhat complex, scenario:
\begin{alltt}
valof\{
  var X := 0;
  var inc is (procedure() do \{ X:=X+1; \})
  valis X
\}
\end{alltt}
the assignment to \q{X} within the \q{inc} procedure is permitted; even though it  side-effect a variable not defined directly within the procedure.
\end{aside}

\subsubsection{Declaring Variables}
\index{variable!declaration}
The type of a \ntRef{Variable} can be declared in an action sequence using a \ntRef{TypeAnnotation} statement prior to the declaration itself:
\begin{alltt}
X has type ref integer;
var X := 3
\end{alltt}

\subsubsection{Type Safety}
A variable declaration is type safe if the type of the variable is the same as the type of the expression giving its value.
\begin{aside}
Of course, it is often the case that the type of a variable is determined from its declaration; so type safety is typically more an issue for other references to the variable identifier than for the variable declaration itself.
\end{aside}

\begin{prooftree}
\AxiomC{\typeprd{E}{Ex}{T}}
\AxiomC{\typeprd{E}{P}{T}}
\BinaryInfC{\typesafe{E}{\q{var} P\ \q{is}\ Ex}}
\end{prooftree}

\begin{prooftree}
\AxiomC{\typeprd{E}{Ex}{T}}
\AxiomC{\typeprd{E}{V}{\q{ref}\ T}}
\BinaryInfC{\typesafe{E}{\q{var}\ V\ \q{:=}\ Ex}}
\end{prooftree}

\subsection{Assignment}
\label{assignment}
\index{action!assignment}
\index{assignment}
The assignment action \q{:=} replaces the contents of a variable with a new value. For example:
\begin{alltt}
Count := Count+3
\end{alltt}
changes the value associated with the variable \q{Count} to \q{Count+3} -- where \q{Count+3} refers to the `old' value of \q{Count}.

There are a number of variations on the basic form of assignment; it is possible to `replace' an element of a \q{list} or an attribute of a record. However, semantically, all the different syntactic forms of assignment have a common root: that of changing a variable to have a different value.

Figures~\vref{assignmentFig}, \vref{recordTargetFig}, and \vref{IndexTargetFig} show the different syntactic forms of an assignment action.

\begin{aside}
Assignment is restricted to replacing the value of a \q{ref}erence typed variable or record field.
\end{aside}

\begin{figure}[htbp]
\begin{eqnarray*}
\ntDef{Assignment}&\arrow&\ntRef{VariableAssignment}\\
&\choice&\ntRef{IndexedAssignment}\\
&\choice&\ntRef{RecordAssignment}\\
\ntDef{VariableAssignment}&\arrow&\ntRef{Variable}\ \q{:=}\ \ntRef{Expression}
\end{eqnarray*}
\caption{Assignment Action}
\label{assignmentFig}
\end{figure}

\subsubsection{Type Safety}
A variable assignment is safe iff the type of the variable is a \q{ref}erence type that is consistent with the expression denoting the variable's new value.

\begin{prooftree}
\AxiomC{\typeprd{E}{V}{\q{ref}\ T}}
\AxiomC{\typeprd{E}{Vl}{T}}
\BinaryInfC{\typesafe{E}{V\q{ := }Vl}}
\end{prooftree}

\subsection{Updating Records}
\label{recordUpdate}
\index{record values!update}
\index{update record values}
An individual field of a record may be updated using the dot-notation on the left hand side of an assignment action -- provided that the type of the field is a \q{ref} type. In effect, assignment to a record field is permitted only if the field was marked as being updateable.

\begin{figure}[htbp]
\begin{eqnarray*}
\ntDef{RecordAssignment}&\arrow&\ntRef{Expression}\,\q{.}\,\ntRef{Identifier}\, \q{:=}\ \ntRef{Expression}\\
\end{eqnarray*}
\caption{Record Assignment}\label{recordTargetFig}
\end{figure}


\subsubsection{Type Safety}
For a record update to be type safe, the field being updated must have \q{ref}erence type.

\begin{prooftree}
\AxiomC{\typeprd{E}{R}{T\sub{R}\ \q{where }T\sub{R}\q{ implements \{}N\q{ has type ref }T\sub{N}\q{\}}}}
\AxiomC{\typeprd{E}{V}{T\sub{N}}}
\BinaryInfC{\typesafe{E}{R\q{.}N\ \q{:=}\ V}}
\end{prooftree}
\begin{aside}
It is \emph{not} necessary for a variable holding the record to be itself re-assignable.
\end{aside}

\subsection{Updating Indexable Collections}
\label{sequenceUpdate}
\index{sequences!update}
\index{update sequences}
An \q{indexable} sequence may be updated using the square index notation on the on the left hand side of an assignment action.

\begin{figure}[htbp]
\begin{eqnarray*}
\ntDef{IndexedAssignment}&\arrow&\ntRef{Expression}\q{[}\ntRef{Expression}\q{]}\q{:=}\ \ntRef{Expression}\\
&\choice&\q{remove}\ \ntRef{Expression}\q{[}\ntRef{Expression}\q{]}
\end{eqnarray*}
\caption{Index Assignment}\label{IndexTargetFig}
\end{figure}

An assignment of the form:
\begin{alltt}
A[ix] := 34
\end{alltt}
is syntactic short-hand for
\begin{alltt}
A := A[with ix->34]
\end{alltt}
which, in turn, is shorthand for:
\begin{alltt}
A := \_set\_indexed(A,ix,34)
\end{alltt}

An assignment of the form:
\begin{alltt}
remove C[ix]
\end{alltt}
means to delete the identified element of the collection and is syntactic shorthand for the assignment:
\begin{alltt}
A := A[without ix]
\end{alltt}
which, in turn, is shorthand for:
\begin{alltt}
A := \_delete\_indexed(A,ix)
\end{alltt}

\begin{aside}
As noted in Section~\vref{indexableContract}, the sequence assignment is not restricted to sequences with \q{integer} indices. The same assignment statement also applies to \q{dictionary} updates.
\end{aside}

\subsubsection{Type Safety}
For an indexable update to be type safe, the left hand side of the assignment must refer to a variable with a \q{ref}erence type -- see Section~\vref{referenceType} -- and whose type implements the \q{indexable} contract -- see Program~\vref{indexableContractDef}.

\begin{prooftree}
\def\defaultHypSeparation{\,}
\AxiomC{\typeprd{E}{s}{\q{ref}\ S\,\q{where indexable over}\ S\ \q{determines}\ \q{(}K\q{,}V\q{)}}}
\AxiomC{\typeprd{E}{k}{K}}
\AxiomC{\typeprd{E}{v}{V}}
\TrinaryInfC{\typesafe{E}{s\q{[}k\q{]}\ \q{:=}\ v}}
\end{prooftree}


\section{Control Flow Actions}
\label{controlFlow}

\subsection{Action Block}
\label{blockAction}
\index{action!block}
\index{block action}
An action block simply consists of a sequence of actions, separated by semicolons and enclosed within the pair of keywords \q{\{} and \q{\}}.

The actions in an action block are executed in sequence.

\begin{figure}[htbp]
\begin{eqnarray*}
\ntDef{ActionBlock}&\arrow&\q{\{}\ \ntRef{Action}\ \q{;}\ldots\q{;} \ntRef{Action}\ \q{\}}
\end{eqnarray*}
\caption{Action Block}
\label{blockActionFig}
\end{figure}

\subsubsection{Scope}
An \ntRef{ActionBlock} represents a \ntRef{Scope}. Any \ntRef{LocalVariable}s that are defined within an \ntRef{ActionBlock} are not defined outside the \ntRef{ActionBlock}.

\subsubsection{Type Safety}
An action block is type safe if each of the actions within it are type safe.

\begin{prooftree}
\AxiomC{\typesafe{E}{A\sub1}}
\AxiomC{\ldots}
\AxiomC{\typesafe{E}{A\subn}}
\TrinaryInfC{\typesafe{E}{\q{\{ }A\sub1;\ldots;A\subn\q{ \}}}}
\end{prooftree}

\subsection{Null Action}
\label{nullAction}
\index{nothing@\q{nothing}}

The \q{nothing} action does nothing. It is type safe by default.

\begin{figure}[htbp]
\begin{eqnarray*}
\ntDef{NullAction}&\arrow&\q{nothing}\ \choice\ \q{\{\}}
\end{eqnarray*}
\caption{Null Action}
\label{nullActionFig}
\end{figure}

\subsection{Let Action}
\label{letActionion}
\index{actions!let action@\q{let} action}
\index{let action@\q{let} action}

A \q{let} action allows an action to be defined in terms of  auxiliary definitions.

\begin{figure}[htbp]
\begin{eqnarray*}
\ntDef{LetAction}&\arrow&\q{let}\ \ntRef{ThetaEnvironment}\ \q{in}\ \ntRef{Action}\\
&\choice&\ntRef{Action}\ \q{using}\ \ntRef{ThetaEnvironment}\\
\end{eqnarray*}
\caption{Let Action}
\label{letActionFig}
\end{figure}

\index{theta environment}
A \q{let} action (or its cousin the \q{using} action) consists of an action that is performed in the enhanced context of a set of auxiliary definition. It is directly analogous to the \ntRef{LetExpression}.


\subsubsection{Type Safety}
The primary safety requirement for a \q{let} action is that the statements that are defined within the body are type consistent. This is the same requirement for any theta environment.

\subsection{Procedure Invocation}
\label{invokeProcedure}
\index{action!invoke procedure}
\index{invoke procedure action}
A procedure invocation is the invocation of an action procedure -- effectively a sub-routine call.


\begin{figure}[htbp]
\begin{eqnarray*}
\ntDef{InvokeAction}&\arrow&\ntRef{Expression}\q{(}\ \ntRef{Expression}\sequence{,}\ntRef{Expression}\ \q{)}
\end{eqnarray*}
\caption{Procedure Invocation}
\label{invokeProcedureFig}
\end{figure}


\subsubsection{Type Safety}
\label{procedureApplyType}
\index{type!procedure invocation}
An action procedure invocation is type safe if the types of the arguments of the application match the argument types of the action procedure.

\begin{prooftree}
\AxiomC{\typeprd{E}{\q{P}}{t\sub{P}}}
\AxiomC{\typeprd{E}{\q{A}}{t\sub{A}}}
\AxiomC{\entail{E}{t\sub{P}\subsume{}t\sub{A}\q{=>()}}}
\TrinaryInfC{\typesafe{E}{\q{P A}}}
\end{prooftree}


\subsubsection{Evaluation Order of Arguments}
\index{procedure invokation!evaluation order}

There is \emph{no} guarantee as to the order of evaluation of arguments to a procedure invocation. In fact, there is no guarantee that a given expression will, in fact, be evaluated. This is similar to the situation with function application.

\begin{aside}
In order to better support parallel execution, it is quite possible that arguments to an procedure invocation are evaluated in parallel; or that their evaluation will be delayed until the value of the argument expression could make a difference to a computation.
\end{aside}

\begin{aside}
In general, the programmer should make the fewest possible assumptions about order of evaluation.
\end{aside}

\subsection{Ignore Action}
\label{ignore}
\index{ignore!ignore action}
\index{ignore action}
\index{action that ignores result}
An \ntRef{IgnoreAction} is an action that simply ignores the value of its \ntRef{Expression} argument.

\begin{figure}[htbp]
\begin{eqnarray*}
\ntDef{IgnoreAction}&\arrow&\q{ignore}\ \ntRef{Expression}
\end{eqnarray*}
\caption{Ignore Action}\label{ignoreActionFig}
\end{figure}

\subsubsection{Type Safety}
An \q{ignore} action is type safe if its ignore expression has a type.

\begin{prooftree}
\AxiomC{\typeprd{E}{Ex}{Tp}}
\UnaryInfC{\typesafe{E}{\q{ignore}\ Ex}}
\end{prooftree}

\begin{aside}
Clearly, the purpose of \q{ignore} is to capture the effect of evaluating an expression. One common purpose of \q{ignore} is to allow a function to be invoked as a procedure call.
\end{aside}

\subsection{For Loop}
\label{forLoop}
\index{action!for loop@\q{for} loop}
\index{for loop action@\q{for} loop action}
\index{loop!for@\q{for}}
A \q{for} loop is used to iterate over the elements of a collection. The collection may be of any of the standard `collection' types:  \q{list}, \q{cons} and \q{dictionary}.

\begin{figure}[htbp]
\begin{eqnarray*}
\ntDef{ForLoop}&\arrow&\q{for}\ \ntRef{Condition}\ \q{do}\ \ntRef{Action}
\end{eqnarray*}
\caption{For Loop}\label{forLoopFig}
\end{figure}
\noindent
For example, the loop:
\begin{alltt}
for ("j",X) in list of [ ("j","s"), ("k","t"), ("j","u") ] do
  logMsg(info,X);
\end{alltt}
results in log messages (see Section~\vref{logMsg}) being printed for \q{"s"} and \q{"u"} (but not for \q{"t"} because \q{("j",X)} does not match against \q{("k","t")}).

A variant of the \q{for} loop allows access to the `index' of the element being processed. For example, in the loop:
\begin{alltt}
for Ix->P in array of ["alpha", "beta", "gamma"] do
  logMsg(info,"P=\$P, index=\$Ix");
\end{alltt}
the variable \q{Ix} is successively bound to the index of the element being processed. 

A \q{for} loop implies a \emph{scope extension}: variables declared in the pattern have their scope extend to the body of the loop. In this case the variable \q{X} introduced in the pattern is available for use in the \q{logMsg} procedure call.

A particularly common case of for loop is the numeric for loop:
\begin{alltt}
for Ix in range(0,10,1) do
  logMsg(info,"\$Ix")
\end{alltt}
This will result in the integers 0 through 9 being displayed on the log.

\subsubsection{Type Safety}
A \q{for} loop is dependent on the \q{iterable} contract (see Section~\vref{iterableContract}; the type safety rules reflect this:
\begin{prooftree}
%\insertBetweenHyps{\hskip 0pt}
\alwaysNoLine
\AxiomC{\typeprd{E}{C}{T\sub{C}\ \q{where iterable over }T\sub{C}\q{ determines (}T\sub{ix}, T\sub{P}\q{)}}}
\def\extraVskip{1ex}
\UnaryInfC{{\typeprd{E}{P}{T\sub{P}} {\hskip 2.5in} \typesafe{E$\,\cup\,$varsin(P)}{Body}}}
\alwaysSingleLine
\UnaryInfC{\typesafe{E}{\q{for }P\ \q{in}\ C\ \q{ do }Body}}
\end{prooftree}

\q{for} loops using the indexed form depend on \q{indexed\_iterable}:
\begin{prooftree}
\alwaysNoLine
\AxiomC{\typeprd{E}{C}{T\sub{C}\ \q{where indexed\_iterable over }T\sub{C}\q{ determines (}T\sub{ix}, T\sub{P}\q{)}}}
\def\extraVskip{1ex}
\UnaryInfC{{\typeprd{E}{P}{T\sub{P}} {\hskip 1in} {\typeprd{E}{Ix}{T\sub{Ix}}}{\hskip 1in}\typesafe{E$\,\cup\,$varsin(P)}{Body}}}
\alwaysSingleLine
\UnaryInfC{\typesafe{E}{\q{for }Ix\q{ -> }\ P\ \q{in}\ C\ \q{ do }Body}}
\end{prooftree}


\subsection{While Loop}
\label{whileLoop}
\index{action!while loop@\q{while} loop}
\index{while loop action@\q{while} loop action}
\index{loop!while@\q{while}}

The \q{while} loop is used to repetitively evaluate a condition. The loop continues execution for so long as the governing \ntRef{Condition} is satisfiable. 

\begin{figure}[htbp]
\begin{eqnarray*}
\ntDef{WhileLoop}&\arrow&\q{while}\ \ntRef{Condition}\ \q{do}\ \ntRef{Action}
\end{eqnarray*}
\caption{While Loop}
   \label{whileLoopFig}
\end{figure}

A \q{while} loop only makes sense if there is a possibility of successive iterations of the body causing a change of state that would make the condition unsatisfiable. A common paradigm for this is the class of \emph{relaxation} algorithms: algorithms that continue until nothing changes:
\begin{alltt}
var done := false;
while not done do\{
  done := true;
  if \ldots then
    done := false;
\}
\end{alltt}

Like the \q{for} loop, a \q{while} loop also implies a scope extension. Variables defined within the governing condition are available for use within the body of the loop.
\begin{aside}
During each iteration of the \q{while} loop, only the first `solution' to the governing \ntRef{Condition} is `used' and can therefore result in bindings of variables.
\end{aside}

\subsubsection{Type Safety}
The governing condition must be \emph{satisfied}. Other than that, a \q{while} loop is type safe if the body is type safe.

\begin{prooftree}
\AxiomC{\typesat{E}{C}}
\AxiomC{\typesafe{E$\,\cup\,$varsin(C)}{Body}}
\BinaryInfC{\typesafe{E}{\q{while }C\q{ do }Body}}
\end{prooftree}

\subsection{Conditional Action}
\label{ifThenElse}
\index{action!conditional action}
\index{conditional action}
\index{if then else@\q{if} \q{then} \q{else} action}

A conditional action is a straightforward \q{if}\ldots\q{then}\ldots\q{else} action: if the governing condition is satisfied the \q{then} branch is taken; otherwise the \q{else} branch is taken. The \q{else} branch is optional in a conditional action; if not present then no action is taken if the condition is not \q{true}.

\begin{figure}[htbp]
\begin{eqnarray*}
\ntDef{ConditionalAction}&\arrow&\q{if}\ \ntRef{Condition}\ \q{then}\ \ntRef{Action}\ [\ \q{else}\ \ntRef{Action}\ ]
\end{eqnarray*}
\caption{Conditional Action}\label{conditionalActionFig}
\end{figure}

\begin{aside}
The `test' part of a conditional action takes the form of a \emph{Condition}. This implies that the test may bind variables -- those variables are in scope within the `then' action but are not in scope for the `else' action.
\end{aside}
\begin{aside}
In general, a condition may be satisfied in many different ways. The conditional action only looks for the `first' way of satisfying the condition.
\end{aside}

For example, we can use a \ntRef{Search} condition to verify that an element is in a collection. The fragment:
\begin{alltt}
if \{name="j"; amount=X\} in Scores then
  \emph{Action}
\end{alltt}
tests to see if there is an entry that matches \q{\{name="j"; amount=X\}} in the \q{Scores} collection; and, if there is, binds the variable \q{X} appropriately within \q{\emph{Action}}.

\subsubsection{Type Safety}
A conditional action is type safe if the condition is safe, and if both the branches are type safe.

\begin{prooftree}
\AxiomC{\typesat{E}{C}}
\AxiomC{\typesafe{E$\,\cup\,$varsin(C)}{Th}}
\AxiomC{\typesafe{E}{El}}
\TrinaryInfC{\typesafe{E}{\q{if }C\q{ then }Th\q{ else }El}}
\end{prooftree}


\subsection{Case Actions}
\label{caseAction}
\index{action!case@\q{case}}
\index{case action@\q{case} action}

A \q{case} action uses a selector expression and a set of action rules to determine which action to perform.
\begin{aside}
As with \q{case} expressions (see Section~\vref{caseExpression}). \q{case} actions are often constructed during the process of compiling other kinds of program.
\end{aside}

\begin{figure}[htbp]
\begin{eqnarray*}
\ntDef{CaseAction}&\arrow&\q{case}\ \ntRef{Expression}\ \q{in}\ \ntRef{CaseActionBody}\\
\ntDef{CaseActionBody}&\arrow&\q{\{}\ntRef{ActionArm}\sequence{\q{;}}\ntRef{ActionArm}\,\q{\}}\\
\ntDef{ActionArm}&\arrow&\ntRef{Pattern}\ \q{do}\ \ntRef{Action}\\
&\choice&\ntRef{Pattern}\ \q{default}\ \q{do}\ \ntRef{Action}
\end{eqnarray*}
\caption{Case Action}
\label{caseActionFig}
\end{figure}

The `selector' expression is evaluated, and then, at most one of the \ntRef{CaseAction}s is selected based on whether the \ntRef{Pattern} matches or not. If one of these does match, then the corresponding \ntRef{Action} on the right hand side is performed.

If none of the \ntRef{ActionArm}'s case patterns match, and if a \q{default} \ntRef{Action} is specified, then that action is performed. If a \q{default} is not specified then \q{nothing} is performed.

Program~\vref{treeWalkProg} shows an example of using a \q{case} action to walk a tree in left-to-right ordering.
\begin{program}
\begin{alltt}
type tree of t is empty or node(tree of t, t, tree of t);

walk has type for all t such that (tree of t, (t)=>())=>()
walk(T,P) \{
  case T in \{
    empty do nothing;
    node(L,Lb,R) do \{
      walk(L,P);
      P(Lb); -- visit the node
      walk(R,P)
    \}
  \}
\};
\end{alltt}
\caption{A Left-to-Right Tree Walk Program\label{treeWalkProg}}
\end{program}


Each \ntRef{ActionArm}'s pattern may introduce variables; these variables are `in scope' only for the corresponding case action.

Optionally, a \q{case} action may have a \q{default} clause. If none of the cases in the \ntRef{CaseActionBody} match then the \q{default} case is performed. If there is no \q{default} clause, then if none of the cases match \q{nothing} is performed -- and execution continues with the next action.


\paragraph{Evaluation Order}
The \ntRef{ActionArm}s in a \ntRef{CaseAction} are tried in the order that they are written -- with the exception of any \q{default} \ntRef{ActionArm} -- which is guaranteed to be attempted only if all others do not apply.

\subsubsection{Type Safety}
The type safety requirements of a \q{case} action are that the types of the patterns of each \ntRef{ActionArm} are the same, and are the same as the selector expression. In addition, the right hand sides of the \ntRef{ActionArm}s should also be consistently typed.

\begin{prooftree}
\AxiomC{\typeprd{E}{S}{T}}
\AxiomC{\typeprd{E}{P\subi}{T}}
\AxiomC{\typesafe{E$\cup{}$varsIn(P\subi)}{A\subi}}
\TrinaryInfC{\typesafe{E}{\q{case}\ S\ \q{in\{}\,P\sub1\ \q{do}\ A\sub1\sequence{;}P\subn\ \q{do}\ A\subn\ \q{\}}}}
\end{prooftree}

In the case that there is a \q{default} clause, then that too must be type safe:
\begin{prooftree}
\AxiomC{\typeprd{E}{S}{T}}
\AxiomC{\typeprd{E}{P\subi}{T}}
\AxiomC{\typesafe{E$\cup{}$varsIn(P\subi)}{A\subi}}
\TrinaryInfC{\typesafe{E}{\q{case}\ S\ \q{in\{}\,P\sub1\ \q{do}\ A\sub1\sequence{;}P\sub{n-1}\ \q{do}\ A\sub{n-1}\, \q{;} P\subn\ \q{default do} A\subn\,\q{\}}}}
\end{prooftree}

\subsection{Valis Action}
\label{valisAction}
\index{action!valis@\q{valis}}
\index{valis action@\q{valis} action}
\index{returning value to \q{valof} expression}

The \q{valis} action determines the value of the nearest textually enclosing  \ntRef{ValueExpression}.

\begin{figure}[htbp]
\begin{eqnarray*}
\ntDef{ValisAction}&\arrow&\q{valis}\ \ntRef{Expression}
\end{eqnarray*}
\caption{Valis Action}
\label{valisActionFig}
\end{figure}

On executing the \q{valis} action, the corresponding \ntRef{ValueExpression} `completes' -- no further actions within the \ntRef{ValueExpression} are executed.


\begin{aside}
The \q{valis} action has special significance within a \ntRef{ComputationExpression}. There the \ntRef{ValisAction} becomes syntactic sugar for an occurrence of the \q{\_encapsulate} function.
\end{aside}

\subsection{Assert Action}
\label{assert}
\index{action!assert action}
\index{assert action}
\index{checking code with assertions}
An \emph{AssertAction} is an action that simply verifies that a particular condition is satisfied. If the assertion is not satisfiable then execution will terminate.

\begin{figure}[htbp]
\begin{eqnarray*}
\ntDef{AssertAction}&\arrow&\q{assert}\ \ntRef{Condition}
\end{eqnarray*}
\caption{Assert Action}\label{assertActionFig}
\end{figure}

\begin{aside}
It is possible to control whether or not assertions are actually executed -- without modifying the source of the program.
\end{aside}


\subsubsection{Type Safety}
An assert action is type safe if the condition is satisfiable.

\begin{prooftree}
\AxiomC{\typesat{E}{C}}
\UnaryInfC{\typesafe{E}{\q{assert}\ C}}
\end{prooftree}

\section{Exceptions and Recovery}
\label{exceptions}
Exceptions represent a way of capturing the non-normal flow of computation. Where a computation may \emph{fail} this may be denoted by an \q{exception} being \q{raise}d during the computation. Raised exceptions may be captured by means of an \q{on abort} handler.
\begin{aside}
Exceptions and abort handling features are an important tool for expressing non-regular flows of computation. However, excessive use of this feature may result in programs that are hard to read.
\end{aside}

\subsection{The \q{exception} Type}
\label{exceptionType}
\index{exception type@\q{exception} type}
\index{type!exception@\q{exception}}
Exceptions and their handling center on the \q{exception} type. When an exception is \q{raise}d, there is an opportunity to communicate a value to the handling code; the \q{exception} is the means by which this is done.

The definition of the \q{exception} type is given in Program~\vref{exceptionDef}.
\begin{program}
\begin{alltt}
type exception is exception(string,any,location)
\end{alltt}
\caption{The definition of the standard \q{exception} type\label{exceptionDef}}
\end{program}

The first element of the \q{exception} constructor is intended to be used as a form of code: it is a string that represents the kind of exception. For internally generated errors, the value of this code is the string \q{"error"}. For user-defined programs, if no value is given to the code then \q{nonString} is used.

The second element of the \q{exception} constructor is an arbitrary exception signal. It is of type \q{any} -- which suggests that it may be any value; however, in most cases, the exception signal is actually a \q{string}.

The third element of the \q{exception} constructor is a \q{location} value. This is typically the source location within the program that gave rise to the exception.

\subsection{Raise Exception}
\label{raiseAction}
\index{raise an exception}
\index{actions!raise exception@\q{raise} exception}
The \q{raise} action is used to cause an exception to be raised.

\begin{figure}[htbp]
\begin{eqnarray*}
\ntDef{RaiseAction}&\arrow&\q{raise}\ \ntRef{Expression}\\
&\choice&\q{raise}\ \ntRef{Expression}\,\q{:}\,\ntRef{Expression}
\end{eqnarray*}
\caption{Raise Expression Action}
\label{raiseActionFig}
\end{figure}
There are two forms to the \q{raise} action's argument: in the first form only the  exception signal is determined. The type of this signal may be any type -- it is represented as an \q{any} value within the \q{exception} value.

In the second form, a \q{string} `code' is given also. This form permits programs to communicate a short flag to the handling mechanism as well as a value. If the code is not given then \q{nonString} is assumed.

A \q{raise} action is equivalent to a call to the standard builtin-function \q{\_raise}. For example, the \q{raise} action:
\begin{alltt}
raise "An exceptional raise"
\end{alltt}
is equivalent to a call to 
\begin{alltt}
_raise(exception(nonString,"An exceptional raise" cast any,__location__))
\end{alltt}
where \q{\_\_location\_\_} is a special \q{location}-valued variable that denotes the source location of the expression itself; see Section~\vref{locationVar}.

The action:
\begin{alltt}
raise "error":34
\end{alltt}
denotes the call:
\begin{alltt}
_raise(exception("error",34 cast any,__location__))
\end{alltt}

A \q{raise} action will cause the current action to abort. If the \q{raise} action is in the dynamic scope of a \ntRef{TryAction} then the protected \ntRef{Action} of that \ntRef{TryAction} is aborted and the recovery \ntRef{Action} is entered.

\begin{aside}
If the \q{raise} action occurs within a computation expression (see Chapter~\vref{computation}, then a \q{raise} action is equivalent to a call to the contract function \q{\_abort}.
\end{aside}

\begin{aside}
The `code' portion of the \q{exception} value is useful in situations where programs must be internationalized. The code may represent an internal code value where the exception value itself is in the form of a presentation \q{string}.
\end{aside}

\subsection{Abort Handling Action}
\label{except}
\index{actions!exception handling}
\index{handling exceptions}
\index{try action@\q{try} action}
The \q{try} \ldots{} \q{on abort} action allows recovery from actions and expressions that cause exceptions.
\begin{figure}[htbp]
\begin{eqnarray*}
\ntDef{TryAction}&\arrow&\q{try}\ \ntRef{Action}\ \q{on abort}\ \ntRef{CaseActionBody}
\end{eqnarray*}
\caption{Try Action}
\label{tryActionFig}
\end{figure}

If an exception is caused during the execution of the protected \ntRef{Action} then the handler in entered. This handler takes the form of the body of a \ntRef{CaseAction} -- i.e., is a sequence of recovery clauses, each of which is a \ntRef{ActionArm}. The pattern part of the recovery clause is matched against the exception value; and the first pattern that matches is used to recover from the exception.

Exceptions are caused either by an error condition -- such as when the equations of a function fail to match a call -- or by an explicit invocation of the \q{raise} action/expression.

For example, in the fragment:
\begin{alltt}
try\{
  A is first(nil); -- Will raise an exception
  logMsg(info,"A is \$A");
\} on abort \{
  E do logMsg(info,"Had exception: \$E");
\}
\end{alltt}
the evaluation of \q{first(nil)} will fail because \q{nil} is empty. As a result, the rest of the \ntRef{Action} it is embedded in is aborted and execution continues with the recovery clause.

\subsubsection{Type Safety}
An \q{try} action is type safe if both arms of the action are safe.

\begin{prooftree}
\AxiomC{\typesafe{E}{P}}
\AxiomC{\typesafe{E}{X}}
\BinaryInfC{\typesafe{E}{\q{try}\ P\ \q{on abort}\ X}}
\end{prooftree}
