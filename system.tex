%!TEX root = reference.tex
\chapter{System Functions}
\label{systemFunctions}

\section{Time Functions}
\label{timeFunctions}
\index{time functions}

\subsection{\q{nanos} -- Time since start}
\begin{alltt}
nanos has type ()=>long
\end{alltt}

The \q{nanos} function returns the number of nanoseconds since the application started.
\begin{aside}
Note that the returned time may not actually be accurate to the nearest nanosecond. The precise accuracy depends on the accuracy of the clock available to the operating system.
\end{aside}
\begin{aside}
For long running applications, the \q{nanos} clock may `roll-over' after approximately 290 years.
\end{aside}


\subsection{\q{sleep} -- Sleep for a period of time}

\begin{alltt}
sleep has type (long)=>()
\end{alltt}
The \q{sleep} action procedure causes the current activity to suspend for a specified number of milliseconds. 
\begin{aside}
Other activities, in particular other \q{actor} activities will continue while this activity is suspended.
\end{aside}

\section{Logging}
An application may log output to standard logging facilities using the \q{logMsg} action procedure.

\subsection{\q{level} -- type}
\index{level type@\q{level} type}
\label{levelType}

The \q{level} type defines a set of logging levels that may be used to indicate the severity of the logged message.
\begin{alltt}
type level is finest
  or finer
  or fine
  or config
  or info
  or warning
  or severe;
\end{alltt}

The different logging levels have an intended interpretation designed to facilitate users of applications manage the type and quantity of logging flow:
\begin{description}
\item[finest] is used for very fine grained logging, typically debugging.
\item[finer] is used for fine grained logging.
\item[fine] is used for reporting of internally significant events within an application.
\item[config] is used to report application configuration events.
\item[info] is used to report important application events.
\item[warning] is used to report a recoverable error condition.
\item[severe] is used to report an unrecoverable error.
\end{description}

\subsection{\q{logMsg} -- log an event}
\label{logMsg}
\begin{alltt}
logMsg has type (level,string)=>()
\end{alltt}

\q{logMsg} is similar to \q{logger}, except that the category is fixed to \q{com.starview.starrules}.

For example, to log an \q{info} level message one can use
\begin{alltt}
logMsg(info,"You need a tune-up")
\end{alltt}

\section{Shell Commands}
\label{shellCommand}

The \q{exec} function allows a \Sr program to execute other processes and to access the return code from the sub-process.

\subsection{\q{exec} -- Execute Sub-Process}
\label{exec}
\index{exec@\q{exec}!sub-process}
\index{sub-process execution}

The \q{exec} function executes a sub-process and returns the integer return code from running the command.

\begin{alltt}
exec has type (string,dictionary of (string,string))=>integer;
\end{alltt}

The first argument is the command line to execute. The format of this, and the valid commands to execute, is system dependent.

The second argument is a \q{dictionary} of environment variables and their values. If the dictionary is empty then the environment variables of the current program are inherited by the sub-process.

The return value from executing the command is returned by \q{exec}. By convention, a return value of zero means that the command succeeded.

\begin{aside}
The \q{exec}'ed command is executed to completion \emph{before} the \q{exec} call returns. If it is desired to execute a sub-process asynchronously then use the \q{spawn} action:
\begin{alltt}
spawn\{ exec("ls -l", dictionary of \{\}) \}
\end{alltt}
\end{aside}


\subsection{\q{getenv} -- Get Environment Variable}
\label{getenv}
\index{environment variable}
\index{getenv@\q{getenv}!environment variable}

\begin{alltt}
getenv has type (string,string)=>string
\end{alltt}

Get the value of an environment variable. The first argument is the name of the variable, the second is a default value should the variable not exist.

\subsection{\q{exit} -- Terminate this Process}
\label{exit}
\index{exit@\q{exit} application}

The \q{exit} action procedure does not return. Instead, it results in the termination of the process performing the \q{exit}.


\begin{alltt}
exit has type (integer)=>()
\end{alltt}

The argument of \q{exit} is used as the value of the process itself. An \q{exit} value of zero implies that the process terminated successfully.
