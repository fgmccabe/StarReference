%!TEX root = reference.tex
\chapter{Introduction}
\label{Introduction}

\Sr is a high-level symbolic programming language oriented to the needs of large-scale high performance processing in modern parallel and distributed computing environments.

\Sr is a `functional-first' language -- in that functions and other programs are first class values. However, it is explicitly not a `pure' language: it has support for updatable variables and structures. However, its bias is definitely in favor of functional programming and in order to get the best value from programming in \Sr, such side-effecting features should be used sparingly.

\Sr is strongly, statically, typed. What this means is that all programs and all values have a single type that is determined by a combination of type inference and explicit type annotations.

While this definitely increases the initial burden of the programmer; we believe that correctness of programs is a long-term productivity gain -- especially for large programs developed by teams of programmers.

The type language is very rich; including polymorphic types, type constraints and higher-rank and higher kinded types. Furthermore, except in cases where higher-ranked types are required, type inference is used extensively to reduce the burden of `type bureaucracy' on programmers.

\Sr is extensible; there are many mechanisms designed to allow extensions to the language to be designed simply and effectively. Using such techniques can significantly ease the burden of writing applications.

\section{About this Reference}
This reference is the language definition of the \Sr language. It is intended to be thorough and as precise as possible in the features discussed. However, where appropriate, we give simple examples as illustrative background to the specification itself.

\subsection{Syntax Rules}
\index{grammar notation}
\index{bnf grammars}

Throughout this document we introduce many syntactic features of the language. We use a variant of traditional BNF grammars to do this. The meta-grammar can be described using itself; as shown in Figure~\vref{metaGrammar}.

\begin{figure}[htbp]
\begin{eqnarray*}
\emph{MetaGrammar}& \arrow& \emph{Production}\sequence{\,}\emph{Production}\\
\emph{Production} &\arrow& \emph{NonTerminal}\ \q{::=}\ \emph{Body}\\
\emph{Production} &\arrowplus&\emph{NonTerminal}\ \q{::+}\ \emph{Body}\\
\emph{Body} &\arrow& \emph{Quoted}\,\choice\,\emph{NonTerminal}\,\choice\,\emph{Choice}\, \choice\,\emph{Optional}\,\choice\,\emph{Sequence}\,\choice\,( \emph{Body} )\\
\emph{Quoted} &\arrow &\q{Characters}\\
\emph{NonTerminal} &\arrow&\emph{Identifier}\\
\emph{Choice}& \arrow &\emph{Body}\sequence{|}\emph{Body}\\
\emph{Optional} &\arrow &\q{[} \emph{Body} \q{]}\\
\emph{Sequence}&\arrow&\emph{Body}\sequence{\rm[\emph{op}]}\emph{Body}\ \choice\,\emph{Body}\sequence{\rm[\emph{op}]}\emph{Body}\plus
\end{eqnarray*}
\caption{Meta-Grammar Used in this Reference}\label{metaGrammar}
\end{figure}

Some grammar combinations are worth explaining as they occur quite frequently and may not be `standard' in BNF-style grammars. For example the rules for actions contain:
\begin{eqnarray*}
\emph{Action}& \arrow&\q{\{}\, \emph{Action}\sequence{;}\emph{Action}\, \q{\}}
\end{eqnarray*}
This grammar rule defines an \emph{Action} as a possibly empty sequence of \emph{Action}s separated by semi-colons and enclosed in braces -- i.e., the classic definition of a block. 

The rule:
\begin{eqnarray*}
\emph{Decimal}& \arrow&\emph{Digit}\sequence{}\emph{Digit}\plus
\end{eqnarray*}
denotes a definition in which there must be at least one occurrence of a \emph{Digit}; in this case there is also no separator between the \emph{Digit}s.

Occasionally, where a non-terminal is not conveniently captured in a single production, later sections will \emph{add} to the definition of the non-terminal. This is signaled with a \arrowplus{} production, as in:
\begin{eqnarray*}
\emph{Expression}&\arrowplus&\emph{ListLiteral}
\end{eqnarray*}
which signals that, in addition to previously defined expressions, a \emph{ListLiteral} is also an \emph{Expression}.

\subsection{Typographical Conventions}
\index{typographical conventions}
Any text on a programming language often has a significant number of examples of programs and program fragments. In this reference, we show these using a \q{typewriter}-like font, often broken out in a display form:
\begin{alltt}
\ldots
P has type integer;
\ldots
\end{alltt}
We use the \q{\ldots} ellipsis to explicitly indicate a fragment of a program that is embedded in a context.
\index{ellipsis}
\index{...}

Occasionally, we have to show a somewhat generic fragment of a program where you, the programmer, are expected to put your own text in. We highlight such areas using an \q{\emph{italicized typewriter}} font:
\begin{alltt}
fn \emph{Args} => \emph{Expr}
\end{alltt}
This kind of notation is intended to suggest that \q{\emph{Args}} and \q{\emph{Expr}} are a kind of \emph{meta-variable} which are intended to be replaced by specific text.

\begin{aside}
Some parts of the text require more careful reading, or represent comments about potential implications of the main text. These notes are highlighted the way this note is, with a \dbend{} symbol.\footnote{Notes which are not really part of the main exposition, but still represent nuggets of wisdom are relegated to footnotes.}
\end{aside}
\begin{aside}
\begin{aside}
Occasionally, there are areas where the programmer may accidentally `trip over' some feature of the language. It seems necessary to mark these with a double \dbend{}\dbend{} symbol.
\end{aside}
\end{aside}